% !TEX TS-program = pdflatex
% !TEX encoding = UTF-8 Unicode

% This is a simple template for a LaTeX document using the "article" class.
% See "book", "report", "letter" for other types of document.

\documentclass[11pt]{article} % use larger type; default would be 10pt

%\usepackage[utf8]{inputenc} % set input encoding (not needed with XeLaTeX)
\usepackage[latin1]{inputenc}
\usepackage[english]{babel}

%%% Examples of Article customizations
% These packages are optional, depending whether you want the features they provide.
% See the LaTeX Companion or other references for full information.

%%% PAGE DIMENSIONS
\usepackage{geometry} % to change the page dimensions
\geometry{a4paper} % or letterpaper (US) or a5paper or....
% \geometry{margin=2in} % for example, change the margins to 2 inches all round
% \geometry{landscape} % set up the page for landscape
%   read geometry.pdf for detailed page layout information

\usepackage{graphicx} % support the \includegraphics command and options

% \usepackage[parfill]{parskip} % Activate to begin paragraphs with an empty line rather than an indent

%%% PACKAGES
\usepackage{booktabs} % for much better looking tables
\usepackage{array} % for better arrays (eg matrices) in maths
\usepackage{paralist} % very flexible & customisable lists (eg. enumerate/itemize, etc.)
\usepackage{verbatim} % adds environment for commenting out blocks of text & for better verbatim
\usepackage{subfig} % make it possible to include more than one captioned figure/table in a single float
% These packages are all incorporated in the memoir class to one degree or another...

%%% HEADERS & FOOTERS
\usepackage{fancyhdr} % This should be set AFTER setting up the page geometry
\pagestyle{fancy} % options: empty , plain , fancy
\renewcommand{\headrulewidth}{0pt} % customise the layout...
\lhead{}\chead{}\rhead{}
\lfoot{}\cfoot{\thepage}\rfoot{}

%%% SECTION TITLE APPEARANCE
\usepackage{sectsty}
\allsectionsfont{\sffamily\mdseries\upshape} % (See the fntguide.pdf for font help)
% (This matches ConTeXt defaults)

%%% ToC (table of contents) APPEARANCE
\usepackage[nottoc,notlof,notlot]{tocbibind} % Put the bibliography in the ToC
\usepackage[titles,subfigure]{tocloft} % Alter the style of the Table of Contents
\renewcommand{\cftsecfont}{\rmfamily\mdseries\upshape}
\renewcommand{\cftsecpagefont}{\rmfamily\mdseries\upshape} % No bold!
\newcommand{\clearemptydoublepage}{\newpage{\pagestyle{plain}\cleardoublepage}}

%%% END Article customizations
\usepackage[hidelinks]{hyperref}
\usepackage{siunitx,multicol}
%\usepackage{natbib}
\usepackage[
    %,style=altlist
    %,hypertoc=true
    %,hyper=true
    ,nonumberlist
    ,acronym=true
    %,toc=true
    ,section=subsection
]{glossaries}

\usepackage{rotating}

\newglossary{dimlessnumber}{dimnu}{dimnb}{Dimensionless Numbers}
\newglossary{greeksymbols}{grsym}{grsbl}{Greek Symbols}
\newglossary{subscripts}{subsc}{subcr}{Subscripts}

\newglossary{example_acronym}{exacr}{excro}{Example Acronyms}
\newglossary{example_symbols}{exsym}{exsbl}{Example Greek Symbols}
\newglossary{example_glossary}{exglo}{exgls}{Example Glossary}


\makeglossaries
\usepackage[xindy]{imakeidx}
\makeindex


%%% The "real" document content comes below...

\title{ThermoCycle Moving Boundary Model}
\author{Adriano Desideri\\
\small Thermodynamics Laboratory\\[-0.8ex]
\small University of Li\`ege\\[-0.8ex]
\small Li\`ege, Belgium\\
\small \texttt{adesideri@ulg.ac.be}\\
\and
Jorrit Wronski\\
\small Department of Mechanical Engineering\\[-0.8ex]
\small Technical University of Denmark\\[-0.8ex]
\small Kgs. Lyngby, Denmark\\
\small \texttt{jowr@mek.dtu.dk}
}

%\date{} % Activate to display a given date or no date (if empty),
         % otherwise the current date is printed 

\begin{document}

% %%%%%%%%%%%%%%%%%%%%%%%%%%%%%%%%%%%%%%%%%%%%%%%%%%%%%%%%%%%%%%%%%%%%%%%
% Dimensionless Groups (based on http://robert.mathmos.net/research/latex/rjwmath.sty)
%  (All two letter command sequences, except for those names already
%   used by LaTeX, and where we need to avoid ambiguity)
% %%%%%%%%%%%%%%%%%%%%%%%%%%%%%%%%%%%%%%%%%%%%%%%%%%%%%%%%%%%%%%%%%%%%%%%
\newcommand{\Bo}{\ensuremath{\operatorname{Bo}}}  % Bond
\newcommand{\Ca}{\ensuremath{\operatorname{Ca}}}  % Capillary
\newcommand{\De}{\ensuremath{\operatorname{De}}}  % Deborah
\newcommand{\Ek}{\ensuremath{\operatorname{E}}}   % Ekman
\newcommand{\Fr}{\ensuremath{\operatorname{Fr}}}  % Froude
\newcommand{\Gr}{\ensuremath{\operatorname{Gr}}}  % Grashof
\newcommand{\Le}{\ensuremath{\operatorname{Le}}}  % Lewis
\newcommand{\Ma}{\ensuremath{\operatorname{Ma}}}  % Mach
\newcommand{\Nu}{\ensuremath{\operatorname{Nu}}}  % Nusselt
\newcommand{\Pe}{\ensuremath{\operatorname{Pe}}}  % Peclet
\newcommand{\Pra}{\ensuremath{\operatorname{Pr}}} % Prandtl
\newcommand{\Ra}{\ensuremath{\operatorname{Ra}}}  % Rayleigh
\newcommand{\Rey}{\ensuremath{\operatorname{Re}}} % Reynolds
\newcommand{\Ri}{\ensuremath{\operatorname{Ri}}}  % Richardson
\newcommand{\Ro}{\ensuremath{\operatorname{Ro}}}  % Rossby
\newcommand{\Sc}{\ensuremath{\operatorname{Sc}}}  % Schmidt
\newcommand{\Sta}{\ensuremath{\operatorname{Sta}}}% Stanton
\newcommand{\Ste}{\ensuremath{\operatorname{S}}}  % Stefan
\newcommand{\Str}{\ensuremath{\operatorname{St}}} % Strouhal
\newcommand{\Ta}{\ensuremath{\operatorname{Ta}}}  % Taylor
\newcommand{\We}{\ensuremath{\operatorname{We}}}  % Weber
\newcommand{\Wi}{\ensuremath{\operatorname{Wi}}}  % Weissenberg
\newcommand{\Wo}{\ensuremath{\operatorname{Wo}}}  % Womersley
%
\newcommand{\Gz}{\ensuremath{\operatorname{Gz}}}  % Graetz
% 
% %%%%%%%%%%%%%%%%%%%%%%%%%%%%%%%%%%%%%%%%%%%%%%%%%%%%%%%%%%%%%%%%%%%%%%%
% Other things
% %%%%%%%%%%%%%%%%%%%%%%%%%%%%%%%%%%%%%%%%%%%%%%%%%%%%%%%%%%%%%%%%%%%%%%%
% Upright pi command 
%\providecommand{\upi}{\mathrm{\pi}}
%
% %%%%%%%%%%%%%%%%%%%%%%%%%%%%%%%%%%%%%%%%%%%%%%%%%%%%%%%%%%%%%%%%%%%%%%%
% Overlap environments for math: http://math.arizona.edu/~aprl/publications/mathclap/
% %%%%%%%%%%%%%%%%%%%%%%%%%%%%%%%%%%%%%%%%%%%%%%%%%%%%%%%%%%%%%%%%%%%%%%%
% For comparison, the existing overlap macros:
% \def\llap#1{\hbox to 0pt{\hss#1}}
% \def\rlap#1{\hbox to 0pt{#1\hss}}
\def\clap#1{\hbox to 0pt{\hss#1\hss}}
\def\mathllap{\mathpalette\mathllapinternal}
\def\mathrlap{\mathpalette\mathrlapinternal}
\def\mathclap{\mathpalette\mathclapinternal}
\def\mathllapinternal#1#2{%
\llap{$\mathsurround=0pt#1{#2}$}}
\def\mathrlapinternal#1#2{%
\rlap{$\mathsurround=0pt#1{#2}$}}
\def\mathclapinternal#1#2{%
\clap{$\mathsurround=0pt#1{#2}$}}



% %%%%%%%%%%%%%%%%%%%%%%%%%%%%%%%%%%%%%%%%%%%%%%%%%%%%%%%%%%%%%%%%%%%%%%%
% Glossaries 
% Read this entry for details: http://www.latex-community.org/know-how/263-glossaries-nomenclature-lists-of-symbols-and-acronyms
% %%%%%%%%%%%%%%%%%%%%%%%%%%%%%%%%%%%%%%%%%%%%%%%%%%%%%%%%%%%%%%%%%%%%%%%

% First we define some entries for the normal glossary
\newglossaryentry{L}{name={\ensuremath{L}},description={Length (\si{\metre})}}

% Then we define some acronyms 
\newacronym{MB}{MB}{moving boundary}

% Next come the dimensionless numbers
\newglossaryentry{Nu}{type=dimlessnumber,
name={\Nu},
description={Nusselt number: $\Nu = \frac{\text{Convection}}{\text{Conduction}} = \frac{hL}{k_f}$},  
%text={\Nu},
first={Nusselt number (\Nu)},
plural={Nusselt numbers},
firstplural={Nusselt numbers (\Nu)}}
%
\newglossaryentry{Rey}{type=dimlessnumber,
name={\Rey},
description={Reynolds number},
%text={\Rey},
first={Reynolds number (\Rey)},
plural={Reynolds numbers},
firstplural={Reynolds numbers (\Rey)}}
%
\newglossaryentry{Pr}{type=dimlessnumber,
name={\Pr},
description={Prandtl number},
%text={\Pr},
first={Prandtl number (\Pr)},
plural={Prandtl numbers},
firstplural={Prandtl numbers (\Pr)}}
%
\newglossaryentry{gr}{type=dimlessnumber,
name={\Gr},
description={Grashof number},
%text={\Gr},
first={Grashof number (\Gr)},
plural={Grashof numbers},
firstplural={Grashof numbers (\Gr)}}
%
\newglossaryentry{ra}{type=dimlessnumber,
name={\Ra},
description={Rayleigh number},
%text={\Ra},
first={Rayleigh number (\Ra)},
plural={Rayleigh numbers},
firstplural={Rayleigh numbers (\Ra)}}
%
\newglossaryentry{gz}{type=dimlessnumber,
name={\Gz},
description={Graetz number},
%text={\Gz},
first={Graetz number (\Gz)},
plural={Graetz numbers},
firstplural={Graetz numbers (\Gz)}}

% And now the Greek symbols
\newglossaryentry{pi}{type=greeksymbols,name={\ensuremath{\pi}},sort=pi,description={ratio of circumference of circle to its diameter}}

% and subscripts
\newglossaryentry{r}{type=subscripts,name={r},description={refrigerant, working fluid}}


% Examples
\newacronym[type=example_acronym]{example_ddye}{D$_{\text{dye}}$}{donor dye, ex. Alexa 488}
\newacronym[type=example_acronym,description={\glslink{example_r0}{F\"{o}rster distance}}]{example_R0}{$R_{0}$}{F\"{o}rster distance}
\newglossaryentry{example_r0}{type=example_glossary,name=\glslink{example_R0}{\ensuremath{R_{0}}},text=F\"{o}rster distance,description={F\"{o}rster distance, where 50\% ...}, sort=R}
\newglossaryentry{example_kdeac}{type=example_glossary,name=\glslink{example_R0}{\ensuremath{k_{DEAC}}},text=$k_{DEAC}$, description={is the rate of deactivation from ... and emission)}, sort=k}
\newglossaryentry{example_pi}{type=example_symbols,name={\ensuremath{\pi}},sort=pi,description={ratio of circumference of circle to its diameter}}


\maketitle

\abstract{The authors present a new moving boundary model that was integrated into the ThermoCycle package written in the Modelica language. Focussing on a seamless integration with existing components, this new component allows to calculate dynamic heat transfer in an efficient and robust way covering the full range of possible operating points in the liquid, two-phase, gas and supercritical domain. A basic validation performed with heat transfer data from two different experiments with evaporators shows that the model is able to reliably predict heat exchanger performance. The flexible implementation allows to compare different heat transfer correlations, which are made freely available as part of the ThermoCycle library. }

\section{Introduction and Motivation}

\Gls{MB} models are established tools to calculate heat exchanger performance in both steady-state and dynamic operation. A fictitious heat transfer channel is split up into different sections and with each section accounting for a different fluid state. In the case of an evaporator the maximum number of sections~\gls{N} is 3 for a)~subcooled, b)~two-phase and c)~superheated state. At higher pressures, the fluid might enter the supercritical state. Hence, there are four different sections out of which a maximum of three can occur simultaneously. The name \glsdesc{MB} is derived from the fact that the interfaces between these sections do not have a fixed spatial position but merely a fixed thermodynamic location depending on the presence of liquid and gaseous fluid, respectively. The actual existence of a certain section and its length are determined based on the fluid state resulting in variable sectioning. A fixed total length superimposes the required boundary condition to calculate the length of each section. 

\Glsdesc{MB} formulations are a good compromise between computational efficiency, robustness and accuracy\cite{Bendapudi2008}. 



\section{Formulation}
In the following sections the moving boundary model is described. In section \ref{sec: Assumptions} the mathematical formulation of the general governing equations describing the fluid flow through a control volume are reported together with the main assumptions adopted for the moving boundary model. Section \ref{sec: OnePhase} and \ref{sec: TwoPhase} report the derivation process for the governing equations for the one phase and the two phase regions respectively. Section \ref{sec: MetalWall} deals with the wall energy balance. In section \ref{sec: SeconadryFluid} the way in which the heat exchanger secondary fluid side is modelled.
% Add a paragraph with the governing equation in a cv : energy mass and momentum balance integrated over length

% Then one phase flow region: mass and energy balance derivation.
% Two phase flow region --> First intruductory section with what a two-phase flow behaviour means 
% - void fraction definition
% - average void fraction definition 
% Mass balance derivation
% Energy balance derivation
% Energy balance wall
% Secondary fluid side
% Add the mean theorem derivation and put it in the appendix
% Add derivation of metal wall equation

\subsection{Governing equations and assumptions}
\label{sec: Assumptions}
The fluid flow through a control volume of the heat exchanger is described with a mathematical formulation of the conservation laws of physics:
\begin{itemize}
\item The mass of the fluid is conserved.
\item The rate of change in momentum equals the sum of the forces on a fluid particles (2nd law of Newton).
\item The rate of change of energy is equal to the sum of the rate of thermal energy addition to and the rate of work done on a fluid particle (Ist law of thermodynamics).
\end{itemize}
The fluid is considered as a continuum. The analysis of the fluid trend is carried out at macroscopic scale length (from 1 \si{\micro m} and larger). As a consequence the fluid molecular structure and motion can be ignored. The evolution of the fluid is described in terms of macroscopic properties i.e. pressure, temperature, density and velocity and their space and time derivatives. The general form of mass energy and momentum balance are reported here under:
 
\begin{equation}
\label{eq:GenMass}
\frac{\partial \rho}{\partial t} + div(\rho \textbf{u})= 0
\end{equation}


\begin{equation}
\label{eq:GenEnergy}
\rho \frac{DE}{Dt} = - div(rho\textbf{u}) + div(k grad T) + S_E
\end{equation}
\begin{equation}
\label{eq:GenMomentum}
General momentum balance
\end{equation}

\subsubsection{Moving boundary model equations}
\label{subsub: MBequations}
In order to derive a low-order model to describe the behaviour of the fluid flowing through an heat exchanger using the moving boundary formulations, the following assumptions are considered:
\begin{enumerate}
\renewcommand{\theenumi}{\roman{enumi}}
\item The Heat exchanger is considered as n 1-dimentional tubes (z-direction) through which the working fluid flows.
\item The tube is cylindrical with a constant cross sectional area.
\item The velocity of the fluid is uniform on the cross sectional area. Homogeneous two-phase flow.
\item Kinetic energy, gravitational forces and viscous stresses are neglected.
\item No work is done on or generated by the control volume.
\item The enthalpy of the fluid is linear in each region of the tube (sub-cooled, two-phase, super-heated)
\item A static momentum balance is considered.
\item Pressure drop through the tube are neglected. Constant pressure.
\item The rate of thermal energy addition due to heat conduction is neglected.
\item the rate of thermal energy addition due to heat convection is considered.
\item Thermal energy accumulation is considered for the metal wall of the tube.
\item Thermal energy conduction in the metal wall is neglected.
\item The secondary fluid is treated as a constant heat capacity fluid.
\end{enumerate}

These assumptions allow to reformulate the conservation equations expressed in equations \ref{eq:GenMass} to \ref{eq:GenMomentum} as:

\begin{equation}
A \cdot \frac{\partial \rho}{\partial t} + \frac{\partial \gls{mdot}}{\partial z}= 0
\label{eq:MassBalance}
\end{equation}

\begin{equation}
A \cdot \frac{\partial(\rho \cdot h - p)}{\partial t} + \frac{\partial ( h \cdot \gls{mdot})}{\partial z} = \gls{Qdot}
\label{eq:EnergyBalance}
\end{equation}

\begin{equation}
p_{\gls{sub:a}} = p_{\gls{sub:b}}
\label{eq:MomentumBalance}
\end{equation}
where A is the cross sectional area of the heat exchanger tube.The derivation process is described in detailed in \cite{KevinDocument}.
In the developed model, the thermodynamic properties are calculated using Coolprop \cite{Bell2013}. The state variables selected are \gls{p} and \gls{hbar}. The convection heat transfer coefficients  on primary side ($\gls{U}_{\gls{sub:pf},\gls{sub:1}}$,$\gls{U}_{\gls{sub:pf},\gls{sub:2}}$,$\gls{U}_{\gls{sub:pf},\gls{sub:3}}$) and secondary side ($\gls{U}_{\gls{sub:sf},\gls{sub:1}}$,$\gls{U}_{\gls{sub:sf},\gls{sub:2}}$,$\gls{U}_{\gls{sub:sf},\gls{sub:3}}$) can be calculated, using appropriate heat transfer model, or imposed as constant\\

 %%%%%%%%%%%%%%%%%%%%%%%%%%%%%%%%%%%%%%%%%%%%%%%%%%%%%%%%%%%%%%%%%%%%%%%%%%
%%%%%%%%%%%%%%%%%%%%%%%%%%%  MASS BALANCE DERIVATION PROCESS %%%%%%%%%%%%%%%%%%%%%%%%%%% %%%%%%%%%%%%%%%%%%%%%%%%%%%%%%%%%%%%%%%%%%%%%%%%%%%%%%%%%%%%%%%%%%%%%%%%%%
\subsection{One phase flow region}
\label{sec: OnePhase}


\subsubsection{Mass balance derivation process}
In order to get the mass balance for one zone of the moving boundary model, we need to integrate equation \ref{eq:MassBalance} over the length of the zone from \gls{sub:a} to {\gls{sub:b}}:

\begin{equation}
A \cdot \int_{\gls{ll}_{\gls{sub:a}}}^{\gls{ll}_{\gls{sub:b}}}  \frac{\partial  \rho }{\partial t} dz + \int_{\gls{ll}_{\gls{sub:a}}}^{\gls{ll}_{\gls{sub:b}}} \frac{\partial \gls{mdot}}{\partial z} dz= 0
\label{eq: MassBalInt}
\end{equation}

Applying Leibniz rule for the first term and solving the integral for the second term of equation \ref{eq: MassBalInt}, results:

\begin{equation}
A \cdot \left[  \frac{d}{dt} \cdot \int_{\gls{ll}_{\gls{sub:a}}}^{\gls{ll}_{\gls{sub:b}}} \rho dz - \rho_{\gls{sub:a}}\cdot \frac{d\gls{ll}_{\gls{sub:a}}}{dt} + \rho_{\gls{sub:b}} \cdot \frac{d\gls{ll}_{\gls{sub:b}}}{dt} \right] = \gls{mdot}_{\gls{sub:a}} - \gls{mdot}_{\gls{sub:b}}
\label{eq: MassAfterLeibniz}
\end{equation}

Solving the  first term of equation \ref{eq: MassAfterLeibniz} results in:
\begin{equation}
 \frac{d}{dt} \cdot \int_{\gls{ll}_{\gls{sub:a}}}^{\gls{ll}_{\gls{sub:b}}} \rho dz =  \frac{d}{dt} \cdot (\overline{\rho} \cdot \gls{ll}) = \overline{\rho} \cdot \frac{d\gls{ll}}{dt} + \gls{ll} \cdot \frac{d\overline{\rho}}{dt}
\label{eq: MassDerDens}
\end{equation}
The bar over the density indicates the mean value of the variable over the considered zone. Substituting equation \ref{eq: MassDerDens} in equation \ref{eq: MassAfterLeibniz} results in:

\begin{equation}
\label{eq:FinalMassBalance}
A \cdot \left[  \overline{\rho} \cdot \frac{d\gls{ll}}{dt} + \gls{ll} \cdot \frac{d\overline{\rho}}{dt} - \rho_{\gls{sub:a}} \cdot \frac{d \gls{ll}_{\gls{sub:a}}}{d \gls{t}} + \rho_{\gls{sub:b}} \cdot \frac{d \gls{ll}_{\gls{sub:b}}}{d \gls{t}} \right] = \gls{mdot}_{\gls{sub:a}} -  \gls{mdot}_{\gls{sub:b}}
\end{equation}

Equation \ref{eq:FinalMassBalance} represents the mass balance for a one phase zone of the heat exchanger.\\ The enthalpy distribution in the zone is considered linear as specified in section \ref{subsub: MBequations}:
\begin{equation}
\label{eq:LinearEnth}
\gls{hbar} = \frac{1}{2} \cdot (h_{\gls{sub:a}} + h_{\gls{sub:b}})
\end{equation}
% Depending on the zone $\rho_{\{aa,bb}}, \gls{ll}_{aa,bb}$ assumes different values summarized in table \ref{Table:Values}.
\begin{equation}
\label{eq:DensOnePhase}
\frac{d \gls{rhobar}}{d \gls{t}} =  \frac{\partial \gls{rhobar}}{\partial p} \cdot \frac{d p}{d \gls{t}} + \frac{\partial \gls{rhobar}}{\partial \gls{hbar}}\cdot \frac{d \gls{hbar}}{d \gls{t}} = 
\frac{\partial \gls{rhobar}}{\partial p} \cdot \frac{d p}{d \gls{t}} +  \frac{1}{2} \cdot \frac{\partial \gls{rhobar}}{\partial \gls{hbar}}  \cdot (\frac{d \gls{h}_{\gls{sub:a}}}{d \gls{t}} + \frac{d \gls{h}_{\gls{sub:b}}}{d \gls{t}})
\end{equation}

where:
\begin{equation}
\label{eq: enthLimits}
\frac{d \gls{h}_{\gls{sub:b}}}{d \gls{t}} =  \frac{\partial \gls{h}_{\gls{sub:l}}}{\partial \gls{p}} \frac{d \gls{p}}{d \gls{t}}  \;\;\;\;\;\; , \;\;\; \frac{d \gls{h}_{\gls{sub:a}}}{d \gls{t}} =  \frac{\partial \gls{h}_{\gls{sub:v}}}{\partial \gls{p}} \frac{d \gls{p}}{d \gls{t}}
\end{equation}
in case of subcooled and superheated zone respectively.

% \frac{1}{2} \cdot \frac{\partial \gls{rhobar}}{\partial \gls{hbar}} \cdot \frac{\partial \gls{h}}{\partial \gls{p}} \frac{d \gls{p}}{d \gls{t}}
%%%%%%%%%%%%%%%%%%%%%%%%%%%%%%%%%%%%%%%%%%%%%%%%%%%%%%%%%%%%%%%%%%%%%%%%%%
%%%%%%%%%%%%%%%%% ENERGY BALANCE DERIVATION PROCESS %%%%%%%%%%%%%%%%%%%%%%
%%%%%%%%%%%%%%%%%%%%%%%%%%%%%%%%%%%%%%%%%%%%%%%%%%%%%%%%%%%%%%%%%%%%%%%%%%
\subsubsection{Energy Balance: derivation process}
Equation \ref{eq:EnergyBalance} is integrated over the length to get the energy balance of one zone of the heat exchanger:

\begin{equation}
\label{eq: EnerBalOneZone}
A \cdot \int_{\gls{ll}_{\gls{sub:a}}}^{\gls{ll}_{\gls{sub:b}}} \frac{\partial(\rho \cdot h)}{\partial t} dz - A \cdot l \cdot \frac{dp}{dt} + \int_{\gls{ll}_{\gls{sub:a}}}^{\gls{ll}_{\gls{sub:b}}} \frac{\partial (h \cdot \gls{mdot})}{\partial z} dz =  \gls{Qdot}
\end{equation}

Applying Leibniz rule for the first integral term and solving the second integral term results in:

\begin{equation}
\label{eq: EnerBalOneZone_2}
A \cdot \left[ \frac{d}{dt} \int_{\gls{ll}_{\gls{sub:a}}}^{\gls{ll}_{\gls{sub:b}}} (\rho \cdot h) dz   +  (\rho_{\gls{sub:a}} h_{\gls{sub:a}}) \cdot \frac{dl_{\gls{sub:a}} }{dt} - (\rho_{\gls{sub:b}} h_{\gls{sub:b}}) \cdot \frac{dl_{\gls{sub:b}} }{dt} \right] - A \cdot l_{\gls{sub:a}} \cdot \frac{dp}{dt}= \gls{mdot}_{\gls{sub:a}} \cdot \gls{h}_{\gls{sub:a}} -  \gls{mdot}_{\gls{sub:b}} \cdot \gls{h}_{\gls{sub:b}} + \gls{Qdot}
\end{equation}

The rate of enthalpy change can be calculated using equation \ref{eq: EnthRate}. Applying the mean-value theorem to get rid of the integral and approximating $\overline{\rho h} = \overline{\rho}\overline{h}$ it results in:

\begin{equation}
 \frac{d}{dt} \int_{\gls{ll}_{\gls{sub:a}}}^{\gls{ll}_{\gls{sub:b}}} (\rho \cdot h) dz = \frac{d}{dt} ( \overline{\rho h} \cdot \gls{ll}) \approx  \frac{d}{dt} ( \overline{\rho} \cdot \overline{h} \cdot \gls{ll}) =  \overline{\rho} \overline{h} \frac{d  \gls{ll}}{dt} + \overline{h} \gls{ll} \frac{d  \overline{\rho}}{dt} + \overline{\rho} \gls{ll} \frac{d  \overline{h}}{dt}
\label{eq: EnthRate}
\end{equation}

Substituting equation \ref{eq: EnthRate} into equation \ref{eq: EnerBalOneZone_2} the energy balance for a one phase zone results in:

\begin{equation}
A \cdot \left[ \overline{\rho} \overline{h} \frac{d  \gls{ll}}{dt} + \overline{h} \gls{ll} \frac{d  \overline{\rho}}{dt} + \overline{\rho} \gls{ll} \frac{d  \overline{h}}{dt} +  (\rho_{\gls{sub:a}} h_{\gls{sub:a}}) \cdot \frac{dl_{\gls{sub:a}} }{dt} - (\rho_{\gls{sub:b}}  h_{\gls{sub:b}}) \cdot \frac{dl_{\gls{sub:b}} }{dt} \right] - A \cdot l_{\gls{sub:a}} \cdot \frac{dp}{dt}= \gls{mdot}_{\gls{sub:a}} \cdot \gls{h}_{\gls{sub:a}} -  \gls{mdot}_{\gls{sub:b}} \cdot \gls{h}_{\gls{sub:b}} + \gls{Qdot}
\end{equation}

which represents the energy balance for the first zone of the heat exchanger. The same procedure can be applied to develop the energy balance for the second and the third zone of the heat exchanger.

%%%%%%%%%%%%%%%%%%%%%%%%%%%%%%%%%%%%%%%%%%%%%%%%%%%%%%%%%%%%%%%%%%%%%%%%%%
%%%%%%%%%%%%%%%%%% TWO PHASE REGION %%%%%%%%%%%%%%%%%%%%%%%
%%%%%%%%%%%%%%%%%%%%%%%%%%%%%%%%%%%%%%%%%%%%%%%%%%%%%%%%%%%%%%%%%%%%%%%%%%
\subsection{Two Phase region}
In this section the mass and energy balance for the two-phase zone are reported. A description of two-phase flow behaviour together with the definition of the primary variables characterising two-phase flow is reported in the appendix. 
\subsubsection{Mass balance}
Assuming homogeneous two-phase flow condition, the average density is calculated based on equation \ref{eq:rhoAveTwoPhase}: 

\begin{equation}
\overline{\rho} = (1-\overline{\gamma}) \rhop  + \overline{\gamma} \rhopp 
\label{eq:rhoAveTwoPhase}
\end{equation}

where $ \overline{\gamma} $ is the average void fraction.
Based on equation \ref{eq:rhoAveTwoPhase}, the rate of mass change for a two-phase flow becomes:

\begin{gather}
 \frac{d}{dt} \int_{\gls{ll}_{\gls{sub:a}}}^{\gls{ll}_{\gls{sub:b}}} \rho  dz = \frac{d}{dt} ( \overline{\rho} \cdot \gls{ll}) = \frac{d}{dt} ( \overline{\rho} \cdot \gls{ll})  \frac{d}{dt} (\gls{ll}\cdot ((1-\overline{\gamma}) \rhop  + \overline{\gamma} \rhopp )) =  \\
= \frac{d\gls{ll}}{dt} \cdot ((1-\overline{\gamma}) \rhop  + \overline{\gamma} \rhopp )) + ll \cdot (( \frac{d \overline{\gamma}}{dt}\rhopp +  \frac{d \rhopp}{dt}  \overline{\gamma} + \frac{d \rhop}{dt} - \overline{\gamma} \frac{d \rhop}{dt} - \rhop \frac{d \overline{\gamma}}{dt}) = \\
= \frac{d\gls{ll}}{dt} \cdot ((1-\overline{\gamma}) \rhop  + \overline{\gamma} \rhopp )) + ll \cdot ((\rhopp - \rhop) \frac{d \overline{\gamma}}{dt} + \overline{\gamma} \frac{d \rhopp}{dp}\frac{d p}{dt} + (1-\overline{\gamma})\frac{d \rhop}{dt}) 
\label{eq:MassFlowRateTwoPhase}
\end{gather}

Substituting equation \ref{eq:MassFlowRateTwoPhase} in equation \ref{eq: MassAfterLeibniz} the mass balance for a two-phase flow results in:

\begin{gather}
A \cdot [ \frac{d\gls{ll}}{dt} \cdot ((1-\overline{\gamma}) \rhop  + \overline{\gamma} \rhopp )) + ll \cdot ((\rhopp - \rhop)\frac{d \overline{\gamma}}{dt} + \overline{\gamma} \frac{d \rhopp}{dp}\frac{d p}{dt} + (1-\overline{\gamma})\frac{d \rhop}{dt}) - \\
- \rho_{\gls{sub:a}}\cdot \frac{d\gls{ll}_{\gls{sub:a}}}{dt} + \rho_{\gls{sub:b}} \cdot \frac{d\gls{ll}_{\gls{sub:b}}}{dt} ] =  \gls{mdot}_{\gls{sub:a}} - \gls{mdot}_{\gls{sub:b}}
\label{eq: MassBalanceTwoPhase}
\end{gather}
\subsubsection{Energy balance}
Considering equation \ref{eq:rhoAveTwoPhase}, the rate of energy change for a two-phase flow fluid results in:
\begin{gather}
\label{eq:RateEnergyTwoPhase}
\frac{d}{dt} \int_{\gls{ll}_{\gls{sub:a}}}^{\gls{ll}_{\gls{sub:b}}} (\rho \cdot h) dz = \frac{d}{dt} ( \overline{\rho h} \cdot \gls{ll}) \approx  \frac{d}{dt} ( \overline{\rho} \cdot \overline{h} \cdot \gls{ll}) = \frac{d}{dt} (\gls{ll} \cdot ((1- \overline{\gamma)}\rhop \hp + \overline{\gamma} \rhopp \hpp)) = \\
\frac{d\gls{ll}}{dt} \cdot ((1-\overline{\gamma}) \rhop \hp  + \overline{\gamma} \rhopp \hpp) + \gls{ll}\cdot(\rhopp \hpp \frac{ d\overline{\gamma}}{dt} + \overline{\gamma} \rhopp \frac{d\hpp}{dt} + \overline{\gamma} \hpp \frac{d\rhopp}{dt}+ \\ + \hp \frac{d \rhop}{dt} + \rhop \frac{d \hp}{dt} - \hp \rhop \frac{d\overline{\gamma}}{dt} - \overline{\gamma} \hp \frac{d \rhop}{dt} - \overline{\gamma} \rhop \frac{d \hp}{dt}) = \\
= \frac{d\gls{ll}}{dt} \cdot ((1-\overline{\gamma}) \rhop \hp  + \overline{\gamma} \rhopp \hpp) + \gls{ll}\cdot((\rhopp \hpp - \rhop \hp ) \frac{d \overline{\gamma}}{dt} + \overline{\gamma}\hpp \frac{\partial \rhopp}{\partial p}\frac{d p}{dt} + \overline{\gamma}\rhopp\frac{\partial \hpp }{dp}\frac{d p}{dt} + \\
+ (1-\overline{\gamma})\hp\frac{\partial \rhop}{dp}\frac{d p}{dt} + (1-\gamma)\rhop\frac{\partial \hp }{\partial p}\frac{d p}{dt})
\label{eq:EnergyRateTwoPhase}
\end{gather}

Substituting equation \ref{eq:EnergyRateTwoPhase}, in equation \ref{eq: EnerBalOneZone_2}, the two-phase flow energy balance results:

\begin{gather}
A \cdot [ \frac{d\gls{ll}}{dt} \cdot ((1-\overline{\gamma}) \rhop \hp  + \overline{\gamma} \rhopp \hpp) + \gls{ll}\cdot((\rhopp \hpp - \rhop \hp ) \frac{d \overline{\gamma}}{dt} + \overline{\gamma}\hpp \frac{\partial \rhopp}{\partial p}\frac{d p}{dt} + \overline{\gamma}\rhopp\frac{\partial \hpp }{dp}\frac{d p}{dt} + \\
+ (1-\overline{\gamma})\hp\frac{\partial \rhop}{dp}\frac{d p}{dt} + (1-\gamma)\rhop\frac{\partial hp }{\partial p}\frac{d p}{dt}) +  (\rho_{\gls{sub:a}} h_{\gls{sub:a}}) \cdot \frac{dl_{\gls{sub:a}} }{dt} - (\rho_{\gls{sub:b}}  h_{\gls{sub:b}}) \cdot \frac{dl_{\gls{sub:b}} }{dt} ] - A \cdot l_{\gls{sub:a}} \cdot \frac{dp}{dt}= \\
= \gls{mdot}_{\gls{sub:a}} \cdot \gls{h}_{\gls{sub:a}} -  \gls{mdot}_{\gls{sub:b}} \cdot \gls{h}_{\gls{sub:b}} + \gls{Qdot}
\label{eq:EnergyBalanceTwoPhase}
\end{gather}

%%%%%%%%%%%%%%%%%%%%%%%%%%%%%%%%%%%%%%%%%%%%%%%%%%%%%%%%%%%%%%%%%%%%%%%%%%
%%%%%%%%%%%%%%%%%% METAL WALL REGION %%%%%%%%%%%%%%%%%%%%%%%
%%%%%%%%%%%%%%%%%%%%%%%%%%%%%%%%%%%%%%%%%%%%%%%%%%%%%%%%%%%%%%%%%%%%%%%%%%
\subsection{Metal wall}
\label{sec: MetalWall}
The energy accumulation in the metal wall can be expressed as:
\begin{equation}
\rho_{\gls{sub:w}} c_{\gls{sub:w}} A_{\gls{sub:w}} \frac{\partial T_{\gls{sub:w}}}{\partial t} =\gls{Qdot}_{\gls{sub:pf}} - \gls{Qdot}_{\gls{sub:sf}}
\end{equation} 

Integrating over the length of one zone it results:

\begin{equation}
\rho_{\gls{sub:w}} c_{\gls{sub:w}} A_{\gls{sub:w}} \int_{\gls{ll}_{\gls{sub:a}}}^{\gls{ll}_{\gls{sub:b}}} \frac{\partial T_{\gls{sub:w}}}{\partial t} dz = \gls{Qdot}_{\gls{sub:pf}} - \gls{Qdot}_{\gls{sub:sf}}
\end{equation}
Applying Leibinz rule:
\begin{equation}
\rho_{\gls{sub:w}} c_{\gls{sub:w}} A_{\gls{sub:w}} [ \frac{d}{d t} \int_{\gls{ll}_{\gls{sub:a}}}^{\gls{ll}_{\gls{sub:b}}} \partial T_{\gls{sub:w}} dz  + T_{\gls{sub:w}}(\gls{ll}_{\gls{sub:b}})\frac{d\gls{ll}_{\gls{sub:b}}}{\partial t}- T_{\gls{sub:w}}(\gls{ll}_{\gls{sub:a}}\frac{d\gls{ll}_{\gls{sub:a}}}{\partial t}) ]= \gls{Qdot}_{\gls{sub:pf}} - \gls{Qdot}_{\gls{sub:sf}}
\end{equation}

Solving the integral the thermal energy balance for the metal wall is:

\begin{equation}
\rho_{\gls{sub:w}} c_{\gls{sub:w}} A_{\gls{sub:w}} [ \frac{d}{d t} \int_{\gls{ll}_{\gls{sub:a}}}^{\gls{ll}_{\gls{sub:b}}} \partial T_{\gls{sub:w}} dz  + T_{\gls{sub:w}}(\gls{ll}_{\gls{sub:b}})\frac{d\gls{ll}_{\gls{sub:b}}}{\partial t}- T_{\gls{sub:w}}(\gls{ll}_{\gls{sub:a}}\frac{d\gls{ll}_{\gls{sub:a}}}{\partial t}) ]= \gls{Qdot}_{\gls{sub:pf}} - \gls{Qdot}_{\gls{sub:sf}}
\end{equation} 
 
%%%%%%%%%%%%%%%%%%%%%%%%%%%%%%%%%%%%%%%%%%%%%%%%%%%%%%%%%%%%%%%%%%%%%%%%%%
%%%%%%%%%%%%%%%%%% SECONDARY FLUID REGION %%%%%%%%%%%%%%%%%%%%%%%
%%%%%%%%%%%%%%%%%%%%%%%%%%%%%%%%%%%%%%%%%%%%%%%%%%%%%%%%%%%%%%%%%%%%%%%%%%
\subsection{Seconday fluid model}
\label{sec: SeconadryFluid}
The secondary fluid model is a steady-state model where the thermal energy transfer with the metal wall is solved using the $\epsilon$-NTU method or the LMTD method.

%%%%%%%%%%%%%%%%%%%%%%%%%%%%%%%%%%%%%%%%%%%%%%%%%%%%%%%%%%%%%%%%%%%%%%%%%%%%%%%%%%%%%%%
%%%%%%%%%%%%%%%%%%%%%%%%%%%%%%%%% HEAT TRANSFER FORMULATION %%%%%%%%%%%%%%%%%%%%%%%%%%%%%%%%%%%%%%%
%%%%%%%%%%%%%%%%%%%%%%%%%%%%%%%%%%%%%%%%%%%%%%%%%%%%%%%%%%%%%%%%%%%%%%%%%%%%%%%%%%%%%%%%%
\subsection{Heat Transfer}

Based on \gls{Nu} from \gls{Rey} and \gls{Pra} for a characteristic length \gls{L}. Angles are usually calculated in radians or \gls{pi}.




%%%%%%%%%%%%%%%%%%%%%%%%%%%%%%%%%%%%%%%%%%%%%%%%%%%%%%%%%%%%%%%%%%%%%%%%%%%%%%%%%%%%%%%
%%%%%%%%%%%%%%%%%%%%%%%%%%%%%%%%% Results and Discussion %%%%%%%%%%%%%%%%%%%%%%%%%%%%%%%%%%%%%%%
%%%%%%%%%%%%%%%%%%%%%%%%%%%%%%%%%%%%%%%%%%%%%%%%%%%%%%%%%%%%%%%%%%%%%%%%%%%%%%%%%%%%%%%%%
\section{Results and Discussion}

Compared to \cite{Kaern2011b}, the model .... 


\cite{Zhang2006}

\cite{Zhang2009}

\section{Conclusion}


\clearemptydoublepage
\appendix{
%%%%%%%%%%%%%%%%%%%%%%%%%%%%%%%%%%%%%%%%%%%%%%%%%%%%%%%%%%%%%%
\section{Two phase flow behaviour}
The two phase flow regime is characterized by the flow patterns that the fluid assumes during evaporation or condensation. The flow configuration depends on the thermodynamic properties of the fluid, the mass flow, the thermal energy flux and the channel geometry. A zone characterized by a certain flow pattern is called flow region.
Visual inspections is the easier approach to detect the flow patterns. Different probes have been developed during the years to accomplish this point. Based on the channel geometry we can distinguish among flow pattern in horizontal and in vertical pipes.
The type of flow configurations are several ecc ecc.
Since the gas and the liquid in a two-phase flow are characterized by different velocity, they can be treated as two distinct continua with an interface where the velocity is the same.
The gas fraction of a two-phase flow can be measured at a local point of the channel by inserting a pipe perpendicular to the channel or by an electrical probe. This would allow to get instantaneous local gas fraction. Averaging over time we can get a local time averaged gas fraction. 
On the other hand an instantaneous volume fraction can be defined as the gas fraction of the total volume considered. Enclosing a pipe section of length Delta z, we can measure the volume fraction. Then letting Delta z goes to zero we get the area averaged gas fraction which is the most adopted definition of void fraction and is then defined as:

\begin{equation}
\overline{\gamma} = \frac{A^{''}}{A}
\end{equation}

\subsection{Average Void Fraction Estimation}\label{subsubsec:avgVoidFrac}

\gls{alphabar}The state variables are pressure and enthalpy and all other properties are functions of those two. The regularly used quantities at the phase boundaries at constant pressure are denoted with $^\prime$ for the liquid phase and $^{\prime\prime}$ for the vapour phase. Hence, the void fraction $\gamma$ can be expressed as a function of vapour fraction $x$ by using the enthalpy-based formulations
\begin{align}
x(p,h)      &= \frac{h-\hp}{\hpp-\hp} \text{ and } \label{eqn:vapourFraction} \\
\gamma(p,h) &= \frac{x \rhop}{ x \rhop + \left( 1 - x \right) \rhopp }\text{. } \label{eqn:voidFraction}
\end{align}
Integrating Eq.~\ref{eqn:voidFraction} over an enthalpy range $\Delta h = h_b - h_a$ allows us to calculate the average void fraction $\overline{\gamma}$ as a function of pressure and the two enthalpies as shown by Bonilla et al.~\cite{Bonilla2015} resulting in 
\begin{gather}
\overline{\gamma}(p,h_a,h_b) = \int_{h_a}^{h_b}\!\!\!\!\!\!\!\!  \gamma(p,h)dh \, \left(h_b - h_a \right)^{-1} \\
= \frac{\rhop{}^2 \left(h_a-h_b\right) + \rhop\rhopp \left(h_b-h_a+\left(\hp-\hpp\right) \ln \left(\frac{\Gamma\left(h_a\right)}{\Gamma\left(h_b\right)}\right)\right)}
{\left(h_a-h_b\right) \left( \rhop-\rhopp \right)^2 } \label{eqn:gammaBar} \\
\text{ with } \Gamma(h) = \rhop (h-\hp) + \rhopp (\hpp-h) \text{. } \label{eqn:Gamma} 
\end{gather}
This $\Gamma(h)$ can be regarded as a density-based fraction of the enthalpy of evaporation 
\begin{equation}
\Gamma(x) = \left( x \rhop + (1-x) \rhopp \right) \left( \hpp-\hp \right) \text{. }
\end{equation}
The $\overline{\gamma}$ obtained from Eq.~\ref{eqn:gammaBar} allows us to calculate the average density $\overline{\rho}$ from 
\begin{equation}
\overline{\rho} = (1-\overline{\gamma}) \rhop + \overline{\gamma} \rhopp \text{. }
\end{equation}
A computationally efficient dynamic model for the two-phase region requires an analytic derivative for the average void fraction. Please see Eq.~\ref{eqn:dgammabardt} through Eq.~\ref{eqn:dgammabardhb} for an expression of $d\overline{\gamma}/dt$ that is explicit in $p$, $h_a$ and $h_b$. 

\paragraph{Average void fraction partial derivatives}

\newcommand{\dhab}{\Delta h_{ab}}
\newcommand{\dhtp}{\Delta h_{tp}}
\newcommand{\drtp}{\Delta \rho_{tp}}
\newcommand{\dhabtp}{\Delta h_{ab,tp}}

\begin{align}
\frac{d\overline{\gamma}}{dt}  =& + \frac{\partial \overline{\gamma}}{\partial p}\frac{dp}{dt} + \frac{\partial \overline{\gamma}}{\partial h_a}\frac{dh_a}{dt} + \frac{\partial \overline{\gamma}}{\partial h_b}\frac{dh_b}{dt} \label{eqn:dgammabardt} \\
%
\frac{\partial \overline{\gamma}}{\partial p}=&+{\frac{\frac{d\rhop}{dp}}{\dhab\drtp^2}\left\lbrace{\dhab\rhop+\rhopp\dhabtp}\right\rbrace}\nonumber\\
&-{\frac{2\rhop\left(\frac{d\rhop}{dp}-\frac{d\rhopp}{dp}\right)}{\dhab\drtp^3}\left\lbrace{\dhab\rhop+\rhopp\dhabtp}\right\rbrace}\nonumber\\
&+{\frac\rhop{\dhab\drtp^2}\left\lbrace{{\dhab{\frac{d\rhop}{dp}}}+\frac{d\rhopp}{dp}\dhabtp}\right.}\nonumber\\
&\quad\left.{+\rhopp\left[{\left({\frac{d\hp}{dp}-\frac{d\hpp}{dp}}\right)\ln\left({G}\right)+\frac{\dhtp}{\Gamma(h_a)}\left({\Theta(h_a)-G\Theta(h_b)}\right)}\right]}\right\rbrace
%
\label{eqn:dgammabardp}\\
%
\frac{\partial \overline{\gamma}}{\partial h_a}=&
-{\frac\rhop{\dhab^{2}\drtp^{2}}\left\lbrace{\dhab\rhop+\rhopp\dhabtp}\right\rbrace}\nonumber\\
&+{\frac\rhop{\dhab\drtp^{2}}\left\lbrace+\rhop+\rhopp\left(-1+\frac{\dhtp\drtp}{\Gamma(h_a)}\right)\right\rbrace}\label{eqn:dgammabardha}\\
%
\frac{\partial \overline{\gamma}}{\partial h_b}=&
+{\frac\rhop{\dhab^{2}\drtp^{2}}\left\lbrace{\dhab\rhop+\rhopp\dhabtp}\right\rbrace}\nonumber\\
&+{\frac\rhop{\dhab\drtp^{2}}\left\lbrace-\rhop+\rhopp\left(+1-\frac{\dhtp\drtp}{\Gamma(h_b)}\right)\right\rbrace}\label{eqn:dgammabardhb}\\
%
\text{ with } 
\dhtp     =&\;\, \hp-\hpp, \quad \drtp = \rhop-\rhopp, \quad \dhab = h_a-h_b, \nonumber\\
\Theta(h) =& \left(h-\hp\right)\frac{d\rhop}{dp}-\rhop\frac{d\hp}{dp}+\left(\hpp-h\right)\frac{d\rhopp}{dp}+\rhopp\frac{d\hpp}{dp}, \nonumber\\
\dhabtp   =& -\dhab+\dhtp\ln\left(G\right), \text{ and } G = \Gamma(h_a) / \Gamma(h_b) \text{. } \nonumber
\end{align}
%\end{sidewaysfigure}



%\begin{sidewaysfigure}
%\textbf{THIS GOES INTO THE APPENDIX}
%\begin{align}
%\frac{d\overline{\gamma}}{dt}  =& + \frac{\partial \overline{\gamma}}{\partial p}\frac{dp}{dt} + \frac{\partial \overline{\gamma}}{\partial h_a}\frac{dh_a}{dt} + \frac{\partial \overline{\gamma}}{\partial h_b}\frac{dh_b}{dt} \label{eqn:dgammabardt} \\
%%
%\frac{\partial \overline{\gamma}}{\partial p} =& + {
%  \frac{\frac{d\rhop}{dp}}{\left( h_a-h_b \right) \left( \rhop-\rhopp \right)^2} 
%  \left\lbrace{ \left( h_a-h_b \right) \rhop + \rhopp \left[{ 
%    h_b-h_a + \left( \hp-\hpp \right) \ln \left({ \frac{\Gamma(h_a)}{\Gamma(h_b)} }\right) 
%  }\right] }\right\rbrace 
%} \nonumber \\ 
%& - {
%  \frac{2 \rhop \left( \frac{d\rhop}{dp} - \frac{d\rhopp}{dp} \right)}{\left( h_a-h_b \right) \left( \rhop-\rhopp \right)^3} 
%  \left\lbrace{ \left( h_a-h_b \right) \rhop + \rhopp \left[{ 
%    h_b-h_a + \left( \hp-\hpp \right) \ln \left({ \frac{\Gamma(h_a)}{\Gamma(h_b)} }\right)
%  }\right] }\right\rbrace 
%} \nonumber \\
%& + {
%  \frac{\rhop}{\left( h_a-h_b \right) \left( \rhop-\rhopp \right)^2} \left\lbrace{
%    { \left( h_a-h_b \right) {\frac{d\rhop}{dp} } } + \frac{d\rhopp}{dp} \left[{ 
%      h_b-h_a + \left( \hp-\hpp \right) \ln \left( \frac{\Gamma(h_a)}{\Gamma(h_b)} \right) 
%    }\right] 
%}\right. } \nonumber \\
%& \quad \left.{ + \rhopp \left[{
%  \left({ \frac {d\hp}{dp} - \frac{d\hpp}{dp} }\right) \ln \left({ \frac{\Gamma(h_a)}{\Gamma(h_b)} }\right)
%  + \frac{\left( \hp-\hpp \right) \left( \Gamma(h_b)  \right) }
%      { \Gamma(h_a) }
%}\right. }\right. \nonumber \\
%& \quad \quad \left.{ \left.{ \cdot \left({
%  \frac{ \left( h_a-\hp \right) \frac{d\rhop}{dp} - \rhop \frac{d\hp}{dp} 
%    + \left( \hpp-h_a \right) \frac{d\rhopp}{dp} + \rhopp \frac{d\hpp}{dp}}
%    {\Gamma(h_b)}
%}\right.}\right.}\right. \nonumber \\ 
%& \quad \quad \quad \left.{ \left.{ \left.{ -
%\frac{ \left( {\rhop}  \left( {h_a}-{\hp} \right) + {\rhopp} \left( {\hpp} - {h_a} \right) \right) 
%          \left( \left( {h_b}-{\hp} \right) {\frac {d{\rhop}}{{d}p}} - {\rhop}  {\frac{d{\hp}}{{d}p}} + 
%               \left( {\hpp}  -{h_b} \right) {\frac {d{\rhopp}}{{d}p}} + {\rhopp}  {\frac {d{\hpp}}{{d}p}}
%          \right) 
%        }{ \left( {\rhop}   \left( {h_b}-{\hp}   \right) +{\rhopp}   \left( {\hpp}  -{h_b} \right) \right) ^{2}}
%        } \right) 
%} \right]  }\right\rbrace
%%
%%
%\label{eqn:dgammabardp} \\
%%
%\frac{\partial \overline{\gamma}}{\partial h_a} =&
%-{\frac{\rhop}{\left(h_a-h_b\right)^{2}\left(\rhop-\rhopp\right)^{2}}\left\lbrace{\left(h_a-h_b\right)\rhop+\rhopp\left[h_b-h_a+\left(\hp-\hpp\right)\ln \left({ \frac{\Gamma(h_a)}{\Gamma(h_b)} }\right)\right]}\right\rbrace} \nonumber \\ 
%&+{\frac{\rhop}{\left(h_a-h_b\right)\left(\rhop-\rhopp\right)^{2}}\left\lbrace+\rhop+\rhopp\left(-1+{\frac{\left(\hp-\hpp\right)\left(\rhop-\rhopp\right)}{\rhop\left(h_a-\hp\right)+\rhopp\left(\hpp-h_a\right)}}\right)\right\rbrace} \label{eqn:dgammabardha} \\
%%
%\frac{\partial \overline{\gamma}}{\partial h_b} =&
%+{\frac{\rhop}{\left(h_a-h_b\right)^{2}\left(\rhop-\rhopp\right)^{2}}\left\lbrace{\left(h_a-h_b\right)\rhop+\rhopp\left[h_b-h_a+\left(\hp-\hpp\right)\ln \left({ \frac{\Gamma(h_a)}{\Gamma(h_b)} }\right)\right]}\right\rbrace} \nonumber \\
%&+{\frac{\rhop}{\left(h_a-h_b\right)\left(\rhop-\rhopp\right)^{2}}\left\lbrace-\rhop+\rhopp\left(+1-{\frac{\left(\hp-\hpp\right)\left(\rhop-\rhopp\right)}{\rhop\left(h_b-\hp\right)+\rhopp\left(\hpp-h_b\right)}}\right)\right\rbrace}\label{eqn:dgammabardhb}
%%
%\end{align}
%\end{sidewaysfigure}

%\begin{equation}
%AL_{\gls{sub:1}} \cdot \left[   \gls{rhobar}_{\gls{sub:1}} \cdot \frac{d \gls{hbar}_{  \gls{sub:1}}}{d \gls{t}} 
%  + \gls{hbar}_{\gls{sub:1}}   \cdot \frac{d \gls{rhobar}_{\gls{sub:1}}}{d \gls{t}}
%  -                                  \frac{d \gls{p}}{d \gls{t}} \right] + A( \overline{\rho}_{\gls{sub:1}} \overline{h}_{\gls{sub:1}} - \rho_{\gls{sub:l}} h_{\gls{sub:l}}   )  \frac{d  L_{\gls{sub:1}}}{dt} = \gls{mdot}_{\gls{sub:a}} \cdot \gls{h}_{\gls{sub:a}} -  \gls{mdot}_{\gls{sub:b}} \cdot \gls{h}_{\gls{sub:b}} + \gls{Qdot}
%\end{equation}
%\begin{sidewaysfigure}
\clearemptydoublepage
\section{Explicit moving boundary formulation for a general evaporator model}


%%%%%%%%%%%%%%%%%%%%%%%%%%%%%%%%%%%%%%%%%%%%%%%%%%%%%%%%%%%%%%%%%%%%%%%%%%%%%%%%%%%%%%%%%%%%%%%%%%%%%%%%
%%%%%%%%%%%%%%%%%%%%%%%%%% SUB COOLED ZONE           - ZONE 1              %%%%%%%%%%%%%%%%%%%%%%%%%%%%%
%%%%%%%%%%%%%%%%%%%%%%%%%%%%%%%%%%%%%%%%%%%%%%%%%%%%%%%%%%%%%%%%%%%%%%%%%%%%%%%%%%%%%%%%%%%%%%%%%%%%%%%%
\begin{center}
{\bf SUB-COOLED ZONE}
\end{center}
{\bf Primary fluid} 
\begin{center}
\textit{Conservation equations}\\
\end{center}


\begin{flushleft}
Mass balance\\
\end{flushleft}
\begin{equation}
\gls{A} \left[\gls{L}_{\gls{sub:1}}  \cdot \frac{d \gls{rhobar}_{\gls{sub:1}}}{d \gls{t}} + (\gls{rhobar}_{\gls{sub:1}} - \rho_{\gls{sub:l}}) \cdot \frac{d \gls{L}_{\gls{sub:1}}}{d \gls{t}}\right] = \gls{mdot}_{\gls{sub:in}} -  \gls{mdot}_{\gls{sub:12}}
\end{equation}

\begin{flushleft}
Energy balance\\
\end{flushleft}
\begin{equation}
\gls{A}\gls{L}_{\gls{sub:1}}\left[
    \gls{rhobar}_{\gls{sub:1}} \cdot \frac{d \gls{hbar}_{  \gls{sub:1}}}{d \gls{t}} 
  + \gls{hbar}_{\gls{sub:1}}   \cdot \frac{d \gls{rhobar}_{\gls{sub:1}}}{d \gls{t}}
  -                                  \frac{d \gls{p}_{     \gls{sub:1}}}{d \gls{t}}
\right] + \gls{A}\left(\gls{rhobar}_{\gls{sub:1}}\gls{hbar}_{\gls{sub:1}} - \rho_{\gls{sub:l}}h_{\gls{sub:l}}\right)\frac{d \gls{L}_{\gls{sub:1}}}{d \gls{t}} = \gls{mdot}_{\gls{sub:in}}  \dot{h}_{\gls{sub:in}} -  \gls{mdot}_\text{A} \dot{h}_\text{A} + \gls{Qdot}_\text{r1}
\end{equation}

\begin{equation}
\frac{d \gls{rhobar}_{\gls{sub:1}}}{d \gls{t}} = \left[ \frac{\partial \gls{rhobar}_{\gls{sub:1}}}{\partial p_{\gls{sub:1}}} + \frac{1}{2} \cdot \frac{\partial \gls{rhobar}_{\gls{sub:1}}}{\partial \gls{hbar}_{\gls{sub:1}}} \cdot \frac{\partial \gls{h}_{\gls{sub:l}}}{\partial \gls{p}_{\gls{sub:1}}}\right] \frac{d \gls{p}_{\gls{sub:1}}}{d \gls{t}} + \frac{1}{2} \cdot \frac{\partial \gls{rhobar}_{\gls{sub:1}}}{\partial \gls{hbar}_{\gls{sub:1}}}  \cdot \frac{d \gls{h}_{\gls{sub:in}}}{d \gls{t}}
\end{equation}


\begin{equation}
\frac{d \gls{hbar}_{\gls{sub:1}}}{d \gls{t}} = \frac{1}{2} \cdot \left[\frac{\partial \gls{hbar}_{\gls{sub:l}}}{\partial p_{\gls{sub:1}}} \cdot \frac{d \gls{p}_{\gls{sub:1}}}{d \gls{t}} + \frac{d \gls{h}_{\gls{sub:in}}}{d \gls{t}}\right]
\end{equation}
\\
\begin{center}
\textit{Constitutive equations}\\
\end{center}

\begin{equation}
\gls{hbar}_{\gls{sub:1}} =  \frac{1}{2}(h_{\gls{sub:in}} + {h}_{\gls{sub:l}})
\end{equation}
\begin{equation}
subcool = setState\_ph(p_{\gls{sub:1}},\gls{hbar}_{\gls{sub:1}})
\end{equation}

\begin{equation}
sat = setSat\_p(p_{\gls{sub:2}})
\end{equation}
\begin{equation}
\rho_{\gls{sub:l}} = bubbleDensity(sat)
\end{equation}
\begin{equation}
h_{\gls{sub:l}} = bubbleEnthalpy(sat)
\end{equation}
\begin{equation}
\frac{\partial \gls{hbar}_{\gls{sub:l}}}{\partial p_{\gls{sub:1}}}= dBubbleEnthalpy\_dPressure(sat)
\end{equation}
\begin{equation}
\frac{\partial \gls{rhobar}_{\gls{sub:1}}}{\partial \gls{hbar}_{\gls{sub:1}}}= density\_derh\_p(subcool)
\end{equation}
\begin{equation}
\frac{\partial \gls{rhobar}_{\gls{sub:1}}}{\partial p_{\gls{sub:1}}}= density\_derp\_h(subcool)
\end{equation}
\begin{equation}
\bar{T}_{\gls{sub:1}} = temperature(subcool)
\end{equation}
\begin{equation}
\gls{Qdot}_\text{r,1} = \pi \text{D} \gls{L}_{\gls{sub:1}} U_\text{pf,1} (T_\text{w,1} - \bar{T}_{\gls{sub:1}})
\end{equation}

%%%%%%%%%%%%% METAL WALL ZONE 1 %%%%%%%%%%%%%%%%%%%%%%%%%%%%%%%%%

{\bf Metal wall}\\
\begin{center}
\textit{Conservation equations}\\
\end{center}

Energy balance:\\

\begin{equation}
C_\text{w}(M_\text{tot} \cdot \frac{\gls{L}_{\gls{sub:1}}}{\gls{L}}) \cdot  \frac{d \text{T}_\text{w,1}}{d t} + \frac{M_\text{tot}}{\gls{L}} (\text{T}_\text{w,1} - \text{T}_\text{w,12})  \frac{d \gls{L}_{\gls{sub:1}}}{d t} = \gls{Qdot}_\text{sf,1} - \gls{Qdot}_\text{r,1}
\end{equation}\\

\begin{center}
\textit{Constitutive equation}\\
\end{center}
\begin{equation}
\text{T}_\text{w,12} = \frac{   \text{T}_\text{w,1} \gls{L}_{\gls{sub:2}}  + \text{T}_\text{w,2}\gls{L}_{\gls{sub:1}}      }{  \gls{L}_{\gls{sub:1}} + \gls{L}_{\gls{sub:2}}         } 
\end{equation}

\begin{equation}
C_\text{w} = const.
\end{equation}

\begin{equation}
M_\text{tot} = const.
\end{equation}

\begin{equation}
L= const.
\end{equation}


%%%%%%%%%%%%%%%%%%%%%%%%%%%%%%%%%%%%%%%%%%%%%%%%%%%%%%%%%%%%%%%%%%%%%%%%%%%%%%%%%%%%%%%%%%%%%%%%%%%%%%%%
%%%%%%%%%%%%%%%%%%%%%%%%%% TWO-PHASE ZONE    -      ZONE 2                 %%%%%%%%%%%%%%%%%%%%%%%%%%%%%
%%%%%%%%%%%%%%%%%%%%%%%%%%%%%%%%%%%%%%%%%%%%%%%%%%%%%%%%%%%%%%%%%%%%%%%%%%%%%%%%%%%%%%%%%%%%%%%%%%%%%%%%

%%%%%%%%%%%%%%%%%%%%%%%%%  ZONE 2 PRIMARY FLUID %%%%%%%%%%%%%%%%%%%%%%%%%%%%
\begin{center}
{\bf TWO-PHASE ZONE}
\end{center}
{\bf Primary fluid}\\
\begin{center}
\textit{Conservation equations}\\
\end{center}
Mass balance:\\
\begin{equation}
\gls{A}\left[\gls{L}_{\gls{sub:2}}  \cdot \frac{d \gls{rhobar}_{\gls{sub:2}}}{d t} + (\gls{rhobar}_{\gls{sub:2}} - \rho_{\gls{sub:v}}) \cdot \frac{d \gls{L}_{\gls{sub:2}}}{d t} + (\rho_{\gls{sub:l}} - \rho_{\gls{sub:v}}) \cdot \frac{d \gls{L}_{\gls{sub:1}}}{d t}\right] =  \gls{mdot}_\text{A} -  \gls{mdot}_{\gls{sub:23}}
\end{equation}

\begin{flushleft}
Energy balance:\\
\end{flushleft}
\begin{equation}
\gls{A}\left[\gls{L}_{\gls{sub:2}} \cdot  \frac{d \gls{rhobar}_{\gls{sub:2}} \gls{hbar}_{\gls{sub:2}}}{d t} +  (\gls{rhobar}_{\gls{sub:2}}\gls{hbar}_{\gls{sub:2}} - \rho_{\gls{sub:v}}h_{\gls{sub:v}}) \cdot \frac{d \gls{L}_{\gls{sub:2}}}{d t}  +  (\rho_{\gls{sub:l}}h_{\gls{sub:l}} - \rho_{\gls{sub:v}}h_{\gls{sub:v}}) \cdot \frac{d \gls{L}_{\gls{sub:1}}}{d t}       -   \gls{L}_{\gls{sub:2}} \cdot  \frac{d p_{\gls{sub:2}}}{d t} \right] =  \gls{mdot}_\text{A}  \dot{h}_\text{A} -  \gls{mdot}_{\gls{sub:23}} \dot{h}_{\gls{sub:23}} + \gls{Qdot}_\text{r,2}
\end{equation}



\begin{equation}
\frac{d \gls{rhobar}_{\gls{sub:2}}}{d t} = ( \frac{\partial \rho_{\gls{sub:v}}}{\partial p_{\gls{sub:2}}} \cdot \bar{\alpha} +  \frac{\partial \rho_{\gls{sub:l}}}{\partial p_{\gls{sub:2}}} \cdot  (1 - \bar{\alpha})) \cdot  \frac{d p_{\gls{sub:2}}}{d t} + (\rho_{\gls{sub:v}} - \rho_{\gls{sub:l}}) \cdot [\frac{\partial \bar{\alpha}}{\partial p_{\gls{sub:2}}} \cdot  \frac{d p_{\gls{sub:2}}}{d t} + \frac{\partial \bar{\alpha}}{\partial h_\text{OUT,alpha}} \cdot \frac{d h_\text{OUT,alpha}}{d t} ]
\end{equation}


\begin{equation} \begin{split}
 \frac{d \gls{rhobar}_{\gls{sub:2}} \gls{hbar}_{\gls{sub:2}}}{d t} &=  [\bar{\alpha} \cdot ( \frac{\partial \rho_{\gls{sub:v}}}{\partial p_{\gls{sub:2}}} \cdot h_{\gls{sub:v}} + \frac{\partial h_{\gls{sub:v}}}{\partial p_{\gls{sub:2}}} \cdot \rho_{\gls{sub:v}} )  + ( 1- \bar{\alpha}) \cdot (  \frac{\partial \rho_{\gls{sub:l}}}{\partial p_{\gls{sub:2}}} \cdot h_{\gls{sub:l}} + \frac{\partial h_{\gls{sub:l}}}{\partial p_{\gls{sub:2}}} \cdot \rho_{\gls{sub:l}} )] \cdot \frac{d p_{\gls{sub:2}}}{d t} \\+
 &(\rho_{\gls{sub:v}}h_{\gls{sub:v}} - \rho_{\gls{sub:l}}h_{\gls{sub:l}}) \cdot  [\frac{\partial \bar{\alpha}}{\partial p_{\gls{sub:2}}} \cdot  \frac{d p_{\gls{sub:2}}}{d t} + \frac{\partial \bar{\alpha}}{\partial h_\text{OUT,alpha}} \cdot \frac{d h_\text{OUT,alpha}}{d t} ]
\end{split}
\end{equation}\\

\begin{center}
\textit{Constitutive equations}
\end{center}
\begin{equation}
\gls{rhobar}_{\gls{sub:2}} = \rho_{\gls{sub:v}} \bar{\alpha} + \rho_{\gls{sub:l}} \cdot (1 - \bar{\alpha})
\end{equation}

\begin{equation}
\gls{rhobar}_{\gls{sub:2}} \gls{hbar}_{\gls{sub:2}}= \rho_{\gls{sub:v}} h_{\gls{sub:v}} \bar{\alpha} + \rho_{\gls{sub:l}} h_{\gls{sub:l}} \cdot (1 - \bar{\alpha})
\end{equation}

\begin{equation}
\rho_{\gls{sub:v}} = dewDensity(sat)
\end{equation}
\begin{equation}
h_{\gls{sub:v}} = dewEnthalpy(sat)
\end{equation}
\begin{equation}
\frac{\partial \gls{hbar}_{\gls{sub:v}}}{\partial p_{\gls{sub:2}}}= dDewEnthalpy\_dPressure(sat)
\end{equation}

\begin{equation}
\bar{T}_{\gls{sub:2}} = temperature(sat)
\end{equation}
\begin{equation}
\gls{Qdot}_\text{r,2} = \pi \text{D} \gls{L}_{\gls{sub:2}} U_\text{pf,2} (T_\text{w,2} - \bar{T}_{\gls{sub:2}})
\end{equation}


%%%%%%%%%%%%%%%%%%%%%%%%%%%%%%%%% ZONE 2  METAL WALL %%%%%%%%%%%%%%%%%%%%%%%%%%%%%%%
\begin{flushleft}
{\bf Metal Wall}\\
\end{flushleft}
\begin{center}
\textit{Conservation equation}
\end{center}
Energy balance:\\
\begin{equation}
C_\text{w} \cdot \frac{M_\text{tot}}{\gls{L}} \cdot [\gls{L}_{\gls{sub:2}} \cdot  \frac{d \text{T}_\text{w,2}}{d t} +  (\text{T}_\text{w,12} - \text{T}_\text{w,23})  \frac{d \gls{L}_{\gls{sub:1}}}{d t} +  (\bar{\text{T}}_\text{w,2} - \text{T}_\text{w,23})  \frac{d \gls{L}_{\gls{sub:2}}}{d t}] = \gls{Qdot}_\text{sf,2} - \gls{Qdot}_\text{r,2}
\end{equation}



%%%%%%%%%%%%%%%%%%%%%%%%%%%%%%%%%%%%%%%%%%%%%%%%%%%%%%%%%%%%%%%%%%%%%%%%%%%%%%%%%%%%%%%%%%%%%%%%%%%%%%%%
%%%%%%%%%%%%%%%%%%%%%%%%%% SUPER-HEATED ZONE    -      ZONE 3              %%%%%%%%%%%%%%%%%%%%%%%%%%%%%
%%%%%%%%%%%%%%%%%%%%%%%%%%%%%%%%%%%%%%%%%%%%%%%%%%%%%%%%%%%%%%%%%%%%%%%%%%%%%%%%%%%%%%%%%%%%%%%%%%%%%%%%
\begin{center}
{\bf SUPER-HEATED ZONE}
\end{center}


{\bf Primary fluid}\\
\begin{center}
\textit{Conservation equations}\\
\end{center}
Mass balance:
\begin{equation}
\text{A} [\gls{L}_{\gls{sub:3}}  \cdot \frac{d \gls{rhobar}_{\gls{sub:3}}}{d t} + (\rho_{\gls{sub:v}} - \gls{rhobar}_{\gls{sub:3}}) \cdot (\frac{d \gls{L}_{\gls{sub:1}}}{d t} + \frac{d \gls{L}_{\gls{sub:2}}}{d t})] = \gls{mdot}_{\gls{sub:23}} -  \gls{mdot}_{\gls{sub:out}}
\end{equation}
\begin{flushleft}
Energy balance:
\end{flushleft}
\begin{equation}
\text{A}\gls{L}_{\gls{sub:3}}[\gls{rhobar}_{\gls{sub:3}} \cdot \frac{d \gls{hbar}_{\gls{sub:3}}}{d t} + \gls{hbar}_{\gls{sub:3}} \cdot \frac{d \gls{rhobar}_{\gls{sub:3}}}{d t}  -  \frac{d p_{\gls{sub:3}}}{d t}] + \text{A}(\rho_{\gls{sub:v}}h_{\gls{sub:v}} - \gls{rhobar}_{\gls{sub:3}}\gls{hbar}_{\gls{sub:3}})\cdot (\frac{d \gls{L}_{\gls{sub:1}}}{d t} + \frac{d \gls{L}_{\gls{sub:2}}}{d t} ) = \gls{mdot}_{\gls{sub:23}}  \dot{h}_{\gls{sub:v}} -  \gls{mdot}_{\gls{sub:out}} \dot{h}_{\gls{sub:out}} + \gls{Qdot}_\text{r,3}
\end{equation}


\begin{equation}
\frac{d \gls{rhobar}_{\gls{sub:3}}}{d t} = [ \frac{\partial \gls{rhobar}_{\gls{sub:3}}}{\partial p_{\gls{sub:3}}} + \frac{1}{2} \cdot \frac{\partial \gls{rhobar}_{\gls{sub:3}}}{\partial \gls{hbar}_{\gls{sub:3}}} \cdot \frac{\partial \gls{hbar}_{\gls{sub:v}}}{\partial p_{\gls{sub:3}}}] \frac{d p_{\gls{sub:3}}}{d t} + \frac{1}{2} \cdot \frac{\partial \gls{rhobar}_{\gls{sub:3}}}{\partial \gls{hbar}_{\gls{sub:3}}}  \cdot \frac{d h_{\gls{sub:out}}}{d t}
\end{equation}


\begin{equation}
\frac{d \gls{hbar}_{\gls{sub:3}}}{d t} = \frac{1}{2} \cdot [\frac{\partial \gls{hbar}_{\gls{sub:v}}}{\partial p_{\gls{sub:3}}} \cdot \frac{d p_{\gls{sub:3}}}{d t} + \frac{d h_{\gls{sub:out}}}{d t}]
\end{equation}\\

\begin{center}
\textit{Constitutive equations}\\
\end{center}

\begin{equation}
vap = setState(p_{\gls{sub:3}},\gls{hbar}_{\gls{sub:3}})
\end{equation}
\begin{equation}
\gls{hbar}_{\gls{sub:3}} = \frac{1}{2}(h_{\gls{sub:v}} - h_{\gls{sub:out}})
\end{equation}
\begin{equation}
\frac{\partial \gls{rhobar}_{\gls{sub:3}}}{\partial \gls{hbar}_{\gls{sub:3}}}= density\_derh\_p(vap)
\end{equation}
\begin{equation}
\frac{\partial \gls{rhobar}_{\gls{sub:3}}}{\partial p_{\gls{sub:3}}}= density\_derp\_h(vap)
\end{equation}
\begin{equation}
\bar{T}_{\gls{sub:3}} = temperature(vap)
\end{equation}
\begin{equation}
\gls{Qdot}_\text{r,3} = \pi \text{D} \gls{L}_{\gls{sub:3}} U_\text{pf,3} (T_\text{w,3} - \bar{T}_{\gls{sub:3}})
\end{equation}

%%%%%%%%%%%%%%%%%%%%%%%%%%%%%%%% ZONE 3 METAL WALL  %%%%%%%%%%%%%%%%%%%%%%%%%%%%%%%%%%%%%%%%%%%%%%
{\bf Metal wall}\\
\begin{center}
\textit{Conservation equations}
\end{center}
Energy balance:
\begin{equation}
C_\text{w}\frac{M_\text{tot}}{L} \cdot [ \gls{L}_{\gls{sub:3}} \cdot  \frac{d \text{T}_\text{w,3}}{d t} + (\text{T}_\text{w,23} - \text{T}_\text{w,3})  (\frac{d \gls{L}_{\gls{sub:1}}}{d t} + \frac{d \gls{L}_{\gls{sub:2}}}{d t})]= \gls{Qdot}_\text{sf,3} - \gls{Qdot}_\text{r,3}
\end{equation}

\begin{center}
\textit{Constitutive equations}
\end{center}

\begin{equation}
\text{T}_\text{w,23} = \frac{   \text{T}_\text{w,3} \gls{L}_{\gls{sub:2}}  + \text{T}_\text{w,2}\gls{L}_{\gls{sub:3}}      }{  \gls{L}_{\gls{sub:2}} + \gls{L}_{\gls{sub:3}}         } 
\end{equation}




%%%%%%%%%%%%%%%%%%%%%%%%%%%%%%%%%%%%%%%%%%%%%%%%
%%%%%%%%%%%%%%%%%% BIBLIOGRAPHY %%%%%%%%%%%%%%%%%%%%%%
%%%%%%%%%%%%%%%%%%%%%%%%%%%%%%%%%%%%%%%%%%%%%%%%

\bibliographystyle{plain}
\bibliography{references}


%%%%%%%%%%%%%%%%%%%%%%%%%%%%%%%%%%%%%%%%%%%%%%%%
%%%%%%%%%%%%%%%%%% NOMENCLATURE %%%%%%%%%%%%%%%%%%%%%%
%%%%%%%%%%%%%%%%%%%%%%%%%%%%%%%%%%%%%%%%%%%%%%%%


%\section{Examples for Nomenclature}
%When used the first time, the full description appears for acronyms: \gls{example_ddye}, \gls{example_R0}, \gls{example_r0}, \gls{example_kdeac}\\
%
%Plural versions can be printed by \glspl{example_r0} and \glspl{example_kdeac}
%Capital letters for sentence beginnings are available as well \Gls{example_ddye}
%
%.. and now we also use the symbols \gls{example_pi}.

%\begin{multicols}{2}
\setglossarysection{section} 
\printglossary[type=main]
\setglossarysection{subsection} 
\printglossary[type=acronym]
\printglossary[type=dimlessnumber]
\printglossary[type=greeksymbols]
\printglossary[type=subscripts]
%\end{multicols}

%\section{Nomenclature}
%\printindex
%\printglossaries




\end{document}
