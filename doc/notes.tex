% !TEX TS-program = pdflatex
% !TEX encoding = UTF-8 Unicode

% This is a simple template for a LaTeX document using the "article" class.
% See "book", "report", "letter" for other types of document.

\documentclass[11pt]{article} % use larger type; default would be 10pt

%\usepackage[utf8]{inputenc} % set input encoding (not needed with XeLaTeX)
\usepackage[latin1]{inputenc}
\usepackage[english]{babel}

%%% Examples of Article customizations
% These packages are optional, depending whether you want the features they provide.
% See the LaTeX Companion or other references for full information.

%%% PAGE DIMENSIONS
\usepackage{geometry} % to change the page dimensions
\geometry{a4paper} % or letterpaper (US) or a5paper or....
% \geometry{margin=2in} % for example, change the margins to 2 inches all round
% \geometry{landscape} % set up the page for landscape
%   read geometry.pdf for detailed page layout information

\usepackage{graphicx} % support the \includegraphics command and options

% \usepackage[parfill]{parskip} % Activate to begin paragraphs with an empty line rather than an indent

%%% PACKAGES
\usepackage{booktabs} % for much better looking tables
\usepackage{array} % for better arrays (eg matrices) in maths
\usepackage{paralist} % very flexible & customisable lists (eg. enumerate/itemize, etc.)
\usepackage{verbatim} % adds environment for commenting out blocks of text & for better verbatim
\usepackage{subfig} % make it possible to include more than one captioned figure/table in a single float
% These packages are all incorporated in the memoir class to one degree or another...

%%% HEADERS & FOOTERS
\usepackage{fancyhdr} % This should be set AFTER setting up the page geometry
\pagestyle{fancy} % options: empty , plain , fancy
\renewcommand{\headrulewidth}{0pt} % customise the layout...
\lhead{}\chead{}\rhead{}
\lfoot{}\cfoot{\thepage}\rfoot{}

%%% SECTION TITLE APPEARANCE
\usepackage{sectsty}
\allsectionsfont{\sffamily\mdseries\upshape} % (See the fntguide.pdf for font help)
% (This matches ConTeXt defaults)

%%% ToC (table of contents) APPEARANCE
\usepackage[nottoc,notlof,notlot]{tocbibind} % Put the bibliography in the ToC
\usepackage[titles,subfigure]{tocloft} % Alter the style of the Table of Contents
\renewcommand{\cftsecfont}{\rmfamily\mdseries\upshape}
\renewcommand{\cftsecpagefont}{\rmfamily\mdseries\upshape} % No bold!

%%% END Article customizations
\usepackage[hidelinks]{hyperref}
\usepackage{siunitx,multicol}
%\usepackage{natbib}
\usepackage[
    %,style=altlist
    %,hypertoc=true
    %,hyper=true
    ,nonumberlist
    ,acronym=true
    %,toc=true
    ,section=subsection
]{glossaries}


\newglossary{dimlessnumber}{dimnu}{dimnb}{Dimensionless Numbers}
\newglossary{greeksymbols}{grsym}{grsbl}{Greek Symbols}
\newglossary{subscripts}{subsc}{subcr}{Subscripts}

\newglossary{example_acronym}{exacr}{excro}{Example Acronyms}
\newglossary{example_symbols}{exsym}{exsbl}{Example Greek Symbols}
\newglossary{example_glossary}{exglo}{exgls}{Example Glossary}


\makeglossaries
\usepackage[xindy]{imakeidx}
\makeindex


%%% The "real" document content comes below...

\title{ThermoCycle Moving Boundary Model}
\author{Adriano Desideri\\
\small Thermodynamics Laboratory\\[-0.8ex]
\small University of Li\`ege\\[-0.8ex]
\small Li\`ege, Belgium\\
\small \texttt{adesideri@ulg.ac.be}\\
\and
Jorrit Wronski\\
\small Department of Mechanical Engineering\\[-0.8ex]
\small Technical University of Denmark\\[-0.8ex]
\small Kgs. Lyngby, Denmark\\
\small \texttt{jowr@mek.dtu.dk}
}

%\date{} % Activate to display a given date or no date (if empty),
         % otherwise the current date is printed 

\begin{document}

% %%%%%%%%%%%%%%%%%%%%%%%%%%%%%%%%%%%%%%%%%%%%%%%%%%%%%%%%%%%%%%%%%%%%%%%
% Dimensionless Groups (based on http://robert.mathmos.net/research/latex/rjwmath.sty)
%  (All two letter command sequences, except for those names already
%   used by LaTeX, and where we need to avoid ambiguity)
% %%%%%%%%%%%%%%%%%%%%%%%%%%%%%%%%%%%%%%%%%%%%%%%%%%%%%%%%%%%%%%%%%%%%%%%
\newcommand{\Bo}{\ensuremath{\operatorname{Bo}}}  % Bond
\newcommand{\Ca}{\ensuremath{\operatorname{Ca}}}  % Capillary
\newcommand{\De}{\ensuremath{\operatorname{De}}}  % Deborah
\newcommand{\Ek}{\ensuremath{\operatorname{E}}}   % Ekman
\newcommand{\Fr}{\ensuremath{\operatorname{Fr}}}  % Froude
\newcommand{\Gr}{\ensuremath{\operatorname{Gr}}}  % Grashof
\newcommand{\Le}{\ensuremath{\operatorname{Le}}}  % Lewis
\newcommand{\Ma}{\ensuremath{\operatorname{Ma}}}  % Mach
\newcommand{\Nu}{\ensuremath{\operatorname{Nu}}}  % Nusselt
\newcommand{\Pe}{\ensuremath{\operatorname{Pe}}}  % Peclet
\newcommand{\Pra}{\ensuremath{\operatorname{Pr}}} % Prandtl
\newcommand{\Ra}{\ensuremath{\operatorname{Ra}}}  % Rayleigh
\newcommand{\Rey}{\ensuremath{\operatorname{Re}}} % Reynolds
\newcommand{\Ri}{\ensuremath{\operatorname{Ri}}}  % Richardson
\newcommand{\Ro}{\ensuremath{\operatorname{Ro}}}  % Rossby
\newcommand{\Sc}{\ensuremath{\operatorname{Sc}}}  % Schmidt
\newcommand{\Sta}{\ensuremath{\operatorname{Sta}}}% Stanton
\newcommand{\Ste}{\ensuremath{\operatorname{S}}}  % Stefan
\newcommand{\Str}{\ensuremath{\operatorname{St}}} % Strouhal
\newcommand{\Ta}{\ensuremath{\operatorname{Ta}}}  % Taylor
\newcommand{\We}{\ensuremath{\operatorname{We}}}  % Weber
\newcommand{\Wi}{\ensuremath{\operatorname{Wi}}}  % Weissenberg
\newcommand{\Wo}{\ensuremath{\operatorname{Wo}}}  % Womersley
%
\newcommand{\Gz}{\ensuremath{\operatorname{Gz}}}  % Graetz
% 
% %%%%%%%%%%%%%%%%%%%%%%%%%%%%%%%%%%%%%%%%%%%%%%%%%%%%%%%%%%%%%%%%%%%%%%%
% Other things
% %%%%%%%%%%%%%%%%%%%%%%%%%%%%%%%%%%%%%%%%%%%%%%%%%%%%%%%%%%%%%%%%%%%%%%%
% Upright pi command 
%\providecommand{\upi}{\mathrm{\pi}}
%
% %%%%%%%%%%%%%%%%%%%%%%%%%%%%%%%%%%%%%%%%%%%%%%%%%%%%%%%%%%%%%%%%%%%%%%%
% Overlap environments for math: http://math.arizona.edu/~aprl/publications/mathclap/
% %%%%%%%%%%%%%%%%%%%%%%%%%%%%%%%%%%%%%%%%%%%%%%%%%%%%%%%%%%%%%%%%%%%%%%%
% For comparison, the existing overlap macros:
% \def\llap#1{\hbox to 0pt{\hss#1}}
% \def\rlap#1{\hbox to 0pt{#1\hss}}
\def\clap#1{\hbox to 0pt{\hss#1\hss}}
\def\mathllap{\mathpalette\mathllapinternal}
\def\mathrlap{\mathpalette\mathrlapinternal}
\def\mathclap{\mathpalette\mathclapinternal}
\def\mathllapinternal#1#2{%
\llap{$\mathsurround=0pt#1{#2}$}}
\def\mathrlapinternal#1#2{%
\rlap{$\mathsurround=0pt#1{#2}$}}
\def\mathclapinternal#1#2{%
\clap{$\mathsurround=0pt#1{#2}$}}



% %%%%%%%%%%%%%%%%%%%%%%%%%%%%%%%%%%%%%%%%%%%%%%%%%%%%%%%%%%%%%%%%%%%%%%%
% Glossaries 
% Read this entry for details: http://www.latex-community.org/know-how/263-glossaries-nomenclature-lists-of-symbols-and-acronyms
% %%%%%%%%%%%%%%%%%%%%%%%%%%%%%%%%%%%%%%%%%%%%%%%%%%%%%%%%%%%%%%%%%%%%%%%

% First we define some entries for the normal glossary
\newglossaryentry{L}{name={\ensuremath{L}},description={Length (\si{\metre})}}

% Then we define some acronyms 
\newacronym{MB}{MB}{moving boundary}

% Next come the dimensionless numbers
\newglossaryentry{Nu}{type=dimlessnumber,
name={\Nu},
description={Nusselt number: $\Nu = \frac{\text{Convection}}{\text{Conduction}} = \frac{hL}{k_f}$},  
%text={\Nu},
first={Nusselt number (\Nu)},
plural={Nusselt numbers},
firstplural={Nusselt numbers (\Nu)}}
%
\newglossaryentry{Rey}{type=dimlessnumber,
name={\Rey},
description={Reynolds number},
%text={\Rey},
first={Reynolds number (\Rey)},
plural={Reynolds numbers},
firstplural={Reynolds numbers (\Rey)}}
%
\newglossaryentry{Pr}{type=dimlessnumber,
name={\Pr},
description={Prandtl number},
%text={\Pr},
first={Prandtl number (\Pr)},
plural={Prandtl numbers},
firstplural={Prandtl numbers (\Pr)}}
%
\newglossaryentry{gr}{type=dimlessnumber,
name={\Gr},
description={Grashof number},
%text={\Gr},
first={Grashof number (\Gr)},
plural={Grashof numbers},
firstplural={Grashof numbers (\Gr)}}
%
\newglossaryentry{ra}{type=dimlessnumber,
name={\Ra},
description={Rayleigh number},
%text={\Ra},
first={Rayleigh number (\Ra)},
plural={Rayleigh numbers},
firstplural={Rayleigh numbers (\Ra)}}
%
\newglossaryentry{gz}{type=dimlessnumber,
name={\Gz},
description={Graetz number},
%text={\Gz},
first={Graetz number (\Gz)},
plural={Graetz numbers},
firstplural={Graetz numbers (\Gz)}}

% And now the Greek symbols
\newglossaryentry{pi}{type=greeksymbols,name={\ensuremath{\pi}},sort=pi,description={ratio of circumference of circle to its diameter}}

% and subscripts
\newglossaryentry{r}{type=subscripts,name={r},description={refrigerant, working fluid}}


% Examples
\newacronym[type=example_acronym]{example_ddye}{D$_{\text{dye}}$}{donor dye, ex. Alexa 488}
\newacronym[type=example_acronym,description={\glslink{example_r0}{F\"{o}rster distance}}]{example_R0}{$R_{0}$}{F\"{o}rster distance}
\newglossaryentry{example_r0}{type=example_glossary,name=\glslink{example_R0}{\ensuremath{R_{0}}},text=F\"{o}rster distance,description={F\"{o}rster distance, where 50\% ...}, sort=R}
\newglossaryentry{example_kdeac}{type=example_glossary,name=\glslink{example_R0}{\ensuremath{k_{DEAC}}},text=$k_{DEAC}$, description={is the rate of deactivation from ... and emission)}, sort=k}
\newglossaryentry{example_pi}{type=example_symbols,name={\ensuremath{\pi}},sort=pi,description={ratio of circumference of circle to its diameter}}


\maketitle

\abstract{The authors present a new moving boundary model that was integrated into the ThermoCycle package written in the Modelica language. Focussing on a seamless integration with existing components, this new component allows to calculate dynamic heat transfer in an efficient and robust way covering the full range of possible operating points in the liquid, two-phase, gas and supercritical domain. A basic validation performed with heat transfer data from two different experiments with evaporators shows that the model is able to reliably predict heat exchanger performance. The flexible implementation allows to compare different heat transfer correlations, which are made freely available as part of the ThermoCycle library. }

\section{Introduction and Motivation}

\Gls{MB} models are established tools to calculate heat exchanger performance in both steady-state and dynamic operation. A fictitious heat transfer channel is split up into different sections and with each section accounting for a different fluid state. In the case of an evaporator the maximum number of sections~\gls{N} is 3 for a)~subcooled, b)~two-phase and c)~superheated state. At higher pressures, the fluid might enter the supercritical state. Hence, there are four different sections out of which a maximum of three can occur simultaneously. The name \glsdesc{MB} is derived from the fact that the interfaces between these sections do not have a fixed spatial position but merely a fixed thermodynamic location depending on the presence of liquid and gaseous fluid, respectively. The actual existence of a certain section and its length are determined based on the fluid state resulting in variable sectioning. A fixed total length superimposes the required boundary condition to calculate the length of each section. 

\Glsdesc{MB} formulations are a good compromise between computational efficiency, robustness and accuracy\cite{Bendapudi2008}. 



\section{Formulation}
\subsection{Assumptions}
\begin{enumerate}
\renewcommand{\theenumi}{\roman{enumi}}
\item The tube is cylindrical with a constant cross sectional area
\item The velocity of the fluid is uniform on the cross sectional area
\item The enthalpy of the fluid is linear in each region of the tube (sub-cooled, two-phase, super-heated)
\item Pressure is considered constant (at least for now)
\item The secondary fluid is treated as a constant heat capacity fluid
\end{enumerate}
\subsection{Equations}
Thermodynamic properties are calculated using Coolprop \cite{Bell2013}. The state variable selected are \gls{p} and \gls{hbar}. The heat transfer coefficients on primary side ($\gls{U}_{\gls{sub:pf},\gls{sub:1}}$,$\gls{U}_{\gls{sub:pf},\gls{sub:2}}$,$\gls{U}_{\gls{sub:pf},\gls{sub:3}}$) and secondary side ($\gls{U}_{\gls{sub:sf},\gls{sub:1}}$,$\gls{U}_{\gls{sub:sf},\gls{sub:2}}$,$\gls{U}_{\gls{sub:sf},\gls{sub:3}}$) are calculated using appropriate heat transfer models. Different model of void fraction \gls{alphabar} have been also implemented.\\
\\

 %%%%%%%%%%%%%%%%%%%%%%%%%%%%%%%%%%%%%%%%%%%%%%%%%%%%%%%%%%%%%%%%%%%%%%%%%%
%%%%%%%%%%%%%%%%%%%%%%%%%%%  MASS BALANCE DERIVATION PROCESS %%%%%%%%%%%%%%%%%%%%%%%%%%% %%%%%%%%%%%%%%%%%%%%%%%%%%%%%%%%%%%%%%%%%%%%%%%%%%%%%%%%%%%%%%%%%%%%%%%%%%
\subsubsection{Mass Balance: derivation process}
The mass balance in the differential form can be written as:
\begin{equation}
\frac{\partial A \cdot \rho}{\partial t} + \frac{\partial \gls{mdot}}{\partial z}= 0
\label{eq: MassBalance}
\end{equation}
To get the mass balance over a certain region (e.g. the sub-cooled region from 0 to $\gls{L}_{\gls{sub:1}}$ ), we need to integrate over z equation \ref{eq: MassBalance}:

\begin{equation}
\int_{0}^{\gls{L}_{\gls{sub:1}}}  \frac{\partial A \cdot \rho_{\gls{sub:1}} }{\partial t} dz + \int_{0}^{\gls{L}_{\gls{sub:1}}} \frac{\partial \gls{mdot}}{\partial z} dz= 0
\label{eq: MassBalInt}
\end{equation}

Applying Leibniz rule for the first term and solving the integral for the second term of equation \ref{eq: MassBalInt}, results:

\begin{equation}
A \cdot \left[  \frac{d}{dt} \cdot \int_{0}^{\gls{L}_{\gls{sub:1}}} \rho_{\gls{sub:1}} dz - \rho_{\gls{sub:1}}(\gls{L}_{\gls{sub:1}}) \cdot \frac{d\gls{L}_{\gls{sub:1}}}{dt} \right] = \gls{mdot}_{\gls{sub:in}} - \gls{mdot}_{\gls{sub:12}}
\label{eq: MassAfterLeibniz}
\end{equation}

Solving the  first term of equation \ref{eq: MassAfterLeibniz} results in:
\begin{equation}
 \frac{d}{dt} \cdot \int_{0}^{\gls{L}_{\gls{sub:1}}} \rho_{\gls{sub:1}} dz =  \frac{d}{dt} \cdot (\overline{\rho}_{\gls{sub:1}} \cdot L_{\gls{sub:1}}) = \overline{\rho}_{\gls{sub:1}} \cdot \frac{dL_{\gls{sub:1}}}{dt} + L_{\gls{sub:1}} \cdot \frac{d\overline{\rho}_{\gls{sub:1}}}{dt}
\label{eq: MassDerDens}
\end{equation}
The bar over the density indicates that a mean value of the zone is considered. Substituting equation \ref{eq: MassDerDens} in equation \ref{eq: MassAfterLeibniz} results in:
\begin{equation}
A \cdot \left[ L_{\gls{sub:1}} \cdot \frac{d\overline{\rho}_{\gls{sub:1}}}{dt}  + (\gls{rhobar}_{\gls{sub:1}} - \rho_{\gls{sub:l}}) \cdot \frac{d \gls{L}_{\gls{sub:1}}}{d \gls{t}}\right] = \gls{mdot}_{\gls{sub:in}} -  \gls{mdot}_{\gls{sub:12}}
\end{equation}
which represents the mass balance for the first zone of the heat exchanger. The same procedure can be applied to develop the mass balance for the second and the third zone of the heat exchanger.

 %%%%%%%%%%%%%%%%%%%%%%%%%%%%%%%%%%%%%%%%%%%%%%%%%%%%%%%%%%%%%%%%%%%%%%%%%%
%%%%%%%%%%%%%%%%%%%%%%%%%%%  ENERGY BALANCE DERIVATION PROCESS %%%%%%%%%%%%%%%%%%%%%%%%%%% %%%%%%%%%%%%%%%%%%%%%%%%%%%%%%%%%%%%%%%%%%%%%%%%%%%%%%%%%%%%%%%%%%%%%%%%%%
\subsubsection{Energy Balance: derivation process}
The energy balance for a zone of the heat exchanger can be written as:
\begin{equation}
A \cdot \frac{\partial(\rho_{\gls{sub:1}} \cdot h_{\gls{sub:1}} - p)}{\partial t} + \frac{\partial ( h_{\gls{sub:1}} \cdot \gls{mdot})}{\partial z} = \gls{Qdot}
\label{eq: EnergyBalance}
\end{equation}

Integrating over length results in:
\begin{equation}
A \cdot \int_{0}^{L_{\gls{sub:1}}} \frac{\partial(\rho_{\gls{sub:1}} \cdot h_{\gls{sub:1}})}{\partial t} dz - A \cdot L_{\gls{sub:1}} \cdot \frac{dp}{dt} + \int_{0}^{L_{\gls{sub:1}}} \frac{\partial (h_{\gls{sub:1}} \cdot \gls{mdot})}{\partial z} dz =  \gls{Qdot}
\label{eq: EnerBalInt}
\end{equation}
where the pressure term is not integrated because is considered constant all over the length. Applying Leibniz rule for the first term of equation \ref{eq: EnerBalInt} results in:

\begin{equation}
A \cdot \left[ \frac{d}{dt} \int_{0}^{L_{\gls{sub:1}}} (\rho_{\gls{sub:1}} \cdot h_{\gls{sub:1}}) dz   -  (\rho \cdot h)_{L_{\gls{sub:1}}} \cdot \frac{dL_{\gls{sub:1}} }{dt}  \right] - A \cdot L_{\gls{sub:1}} \cdot \frac{dp}{dt}= \gls{mdot}_{\gls{sub:in}} \cdot \gls{h}_{\gls{sub:in}} -  \gls{mdot}_{\gls{sub:12}} \cdot \gls{h}_{\gls{sub:12}} + \gls{Qdot}
\label{eq: EnerBqlLeibniz}
\end{equation}

Solving the integral for the first term of equation results in

\begin{equation}
 \frac{d}{dt} \int_{0}^{L_{\gls{sub:1}}} (\rho_{\gls{sub:1}} \cdot h_{\gls{sub:1}}) dz = \frac{d}{dt} ( \overline{\rho}_{\gls{sub:1}} \cdot \overline{h}_{\gls{sub:1}} \cdot L_{\gls{sub:1}}) =\overline{\rho}_{\gls{sub:1}} \overline{h}_{\gls{sub:1}} \frac{d  L_{\gls{sub:1}}}{dt} + \overline{h}_{\gls{sub:1}} L_{\gls{sub:1}} \frac{d  \overline{\rho}_{\gls{sub:1}}}{dt} + \overline{\rho}_{\gls{sub:1}}L_{\gls{sub:1}} \frac{d  \overline{h}_{\gls{sub:1}}}{dt}
\label{eq: EntDensDer}
\end{equation}

Substituting equation \ref{eq: EntDensDer} into equation \ref{eq: EnerBqlLeibniz} and arranging the terms results in:


\begin{equation}
AL_{\gls{sub:1}} \cdot \left[   \gls{rhobar}_{\gls{sub:1}} \cdot \frac{d \gls{hbar}_{  \gls{sub:1}}}{d \gls{t}} 
  + \gls{hbar}_{\gls{sub:1}}   \cdot \frac{d \gls{rhobar}_{\gls{sub:1}}}{d \gls{t}}
  -                                  \frac{d \gls{p}}{d \gls{t}} \right] + A( \overline{\rho}_{\gls{sub:1}} \overline{h}_{\gls{sub:1}} - \rho_{\gls{sub:l}} h_{\gls{sub:l}}   )  \frac{d  L_{\gls{sub:1}}}{dt} = \gls{mdot}_{\gls{sub:in}} \cdot \gls{h}_{\gls{sub:in}} -  \gls{mdot}_{\gls{sub:12}} \cdot \gls{h}_{\gls{sub:12}} + \gls{Qdot}
\end{equation}

which represents the energy balance for the first zone of the heat exchanger. The same procedure can be applied to develop the energy balance for the second and the third zone of the heat exchanger.


\begin{center}
{\bf SUB-COOLED ZONE}
\end{center}
{\bf Primary fluid} 
\begin{center}
\textit{Conservation equations}\\
\end{center}


\begin{flushleft}
Mass balance\\
\end{flushleft}
\begin{equation}
\gls{A} \left[\gls{L}_{\gls{sub:1}}  \cdot \frac{d \gls{rhobar}_{\gls{sub:1}}}{d \gls{t}} + (\gls{rhobar}_{\gls{sub:1}} - \rho_{\gls{sub:l}}) \cdot \frac{d \gls{L}_{\gls{sub:1}}}{d \gls{t}}\right] = \gls{mdot}_{\gls{sub:in}} -  \gls{mdot}_{\gls{sub:12}}
\end{equation}

\begin{flushleft}
Energy balance\\
\end{flushleft}
\begin{equation}
\gls{A}\gls{L}_{\gls{sub:1}}\left[
    \gls{rhobar}_{\gls{sub:1}} \cdot \frac{d \gls{hbar}_{  \gls{sub:1}}}{d \gls{t}} 
  + \gls{hbar}_{\gls{sub:1}}   \cdot \frac{d \gls{rhobar}_{\gls{sub:1}}}{d \gls{t}}
  -                                  \frac{d \gls{p}_{     \gls{sub:1}}}{d \gls{t}}
\right] + \gls{A}\left(\gls{rhobar}_{\gls{sub:1}}\gls{hbar}_{\gls{sub:1}} - \rho_{\gls{sub:l}}h_{\gls{sub:l}}\right)\frac{d \gls{L}_{\gls{sub:1}}}{d \gls{t}} = \gls{mdot}_{\gls{sub:in}}  \dot{h}_{\gls{sub:in}} -  \gls{mdot}_\text{A} \dot{h}_\text{A} + \gls{Qdot}_\text{r1}
\end{equation}

\begin{equation}
\frac{d \gls{rhobar}_{\gls{sub:1}}}{d \gls{t}} = \left[ \frac{\partial \gls{rhobar}_{\gls{sub:1}}}{\partial p_{\gls{sub:1}}} + \frac{1}{2} \cdot \frac{\partial \gls{rhobar}_{\gls{sub:1}}}{\partial \gls{hbar}_{\gls{sub:1}}} \cdot \frac{\partial \gls{h}_{\gls{sub:l}}}{\partial \gls{p}_{\gls{sub:1}}}\right] \frac{d \gls{p}_{\gls{sub:1}}}{d \gls{t}} + \frac{1}{2} \cdot \frac{\partial \gls{rhobar}_{\gls{sub:1}}}{\partial \gls{hbar}_{\gls{sub:1}}}  \cdot \frac{d \gls{h}_{\gls{sub:in}}}{d \gls{t}}
\end{equation}


\begin{equation}
\frac{d \gls{hbar}_{\gls{sub:1}}}{d \gls{t}} = \frac{1}{2} \cdot \left[\frac{\partial \gls{hbar}_{\gls{sub:l}}}{\partial p_{\gls{sub:1}}} \cdot \frac{d \gls{p}_{\gls{sub:1}}}{d \gls{t}} + \frac{d \gls{h}_{\gls{sub:in}}}{d \gls{t}}\right]
\end{equation}
\\
\begin{center}
\textit{Constitutive equations}\\
\end{center}

\begin{equation}
\gls{hbar}_{\gls{sub:1}} =  \frac{1}{2}(h_{\gls{sub:in}} + {h}_{\gls{sub:l}})
\end{equation}
\begin{equation}
subcool = setState\_ph(p_{\gls{sub:1}},\gls{hbar}_{\gls{sub:1}})
\end{equation}

\begin{equation}
sat = setSat\_p(p_{\gls{sub:2}})
\end{equation}
\begin{equation}
\rho_{\gls{sub:l}} = bubbleDensity(sat)
\end{equation}
\begin{equation}
h_{\gls{sub:l}} = bubbleEnthalpy(sat)
\end{equation}
\begin{equation}
\frac{\partial \gls{hbar}_{\gls{sub:l}}}{\partial p_{\gls{sub:1}}}= dBubbleEnthalpy\_dPressure(sat)
\end{equation}
\begin{equation}
\frac{\partial \gls{rhobar}_{\gls{sub:1}}}{\partial \gls{hbar}_{\gls{sub:1}}}= density\_derh\_p(subcool)
\end{equation}
\begin{equation}
\frac{\partial \gls{rhobar}_{\gls{sub:1}}}{\partial p_{\gls{sub:1}}}= density\_derp\_h(subcool)
\end{equation}
\begin{equation}
\bar{T}_{\gls{sub:1}} = temperature(subcool)
\end{equation}
\begin{equation}
\gls{Qdot}_\text{r,1} = \pi \text{D} \gls{L}_{\gls{sub:1}} U_\text{pf,1} (T_\text{w,1} - \bar{T}_{\gls{sub:1}})
\end{equation}

%%%%%%%%%%%%% METAL WALL ZONE 1 %%%%%%%%%%%%%%%%%%%%%%%%%%%%%%%%%

{\bf Metal wall}\\
\begin{center}
\textit{Conservation equations}\\
\end{center}

Energy balance:\\

\begin{equation}
C_\text{w}(M_\text{tot} \cdot \frac{\gls{L}_{\gls{sub:1}}}{\gls{L}}) \cdot  \frac{d \text{T}_\text{w,1}}{d t} + \frac{M_\text{tot}}{\gls{L}} (\text{T}_\text{w,1} - \text{T}_\text{w,12})  \frac{d \gls{L}_{\gls{sub:1}}}{d t} = \gls{Qdot}_\text{sf,1} - \gls{Qdot}_\text{r,1}
\end{equation}\\

\begin{center}
\textit{Constitutive equation}\\
\end{center}
\begin{equation}
\text{T}_\text{w,12} = \frac{   \text{T}_\text{w,1} \gls{L}_{\gls{sub:2}}  + \text{T}_\text{w,2}\gls{L}_{\gls{sub:1}}      }{  \gls{L}_{\gls{sub:1}} + \gls{L}_{\gls{sub:2}}         } 
\end{equation}

\begin{equation}
C_\text{w} = const.
\end{equation}

\begin{equation}
M_\text{tot} = const.
\end{equation}

\begin{equation}
L= const.
\end{equation}


%%%%%%%%%%%%%%%%%%%%%%%%%%%%%%%%%%%%%%%%%%%%%%%%%%%%%%%%%%%%%%%%%%%%%%%%%%%%%
%%%%%%%%%%%%%%%%%%%%% TWO - PHASE ZONE (ZONE 2) %%%%%%%%%%%%%%%%%%%%%%%%%%%%%%%%%%%%%%%%%%%%%%%%
%%%%%%%%%%%%%%%%%%%%%%%%%%%%%%%%%%%%%%%%%%%%%%%%%%%%%%%%%%%%%%%%%%%%%%%%%%%%%



%%%%%%%%%%%%%%%%%%%%%%%%%  ZONE 2 PRIMARY FLUID %%%%%%%%%%%%%%%%%%%%%%%%%%%%
\begin{center}
{\bf TWO-PHASE ZONE}
\end{center}
{\bf Primary fluid}\\
\begin{center}
\textit{Conservation equations}\\
\end{center}
Mass balance:\\
\begin{equation}
\gls{A}\left[\gls{L}_{\gls{sub:2}}  \cdot \frac{d \gls{rhobar}_{\gls{sub:2}}}{d t} + (\gls{rhobar}_{\gls{sub:2}} - \rho_{\gls{sub:v}}) \cdot \frac{d \gls{L}_{\gls{sub:2}}}{d t} + (\rho_{\gls{sub:l}} - \rho_{\gls{sub:v}}) \cdot \frac{d \gls{L}_{\gls{sub:1}}}{d t}\right] =  \gls{mdot}_\text{A} -  \gls{mdot}_{\gls{sub:23}}
\end{equation}

\begin{flushleft}
Energy balance:\\
\end{flushleft}
\begin{equation}
\gls{A}\left[\gls{L}_{\gls{sub:2}} \cdot  \frac{d \gls{rhobar}_{\gls{sub:2}} \gls{hbar}_{\gls{sub:2}}}{d t} +  (\gls{rhobar}_{\gls{sub:2}}\gls{hbar}_{\gls{sub:2}} - \rho_{\gls{sub:v}}h_{\gls{sub:v}}) \cdot \frac{d \gls{L}_{\gls{sub:2}}}{d t}  +  (\rho_{\gls{sub:l}}h_{\gls{sub:l}} - \rho_{\gls{sub:v}}h_{\gls{sub:v}}) \cdot \frac{d \gls{L}_{\gls{sub:1}}}{d t}       -   \gls{L}_{\gls{sub:2}} \cdot  \frac{d p_{\gls{sub:2}}}{d t} \right] =  \gls{mdot}_\text{A}  \dot{h}_\text{A} -  \gls{mdot}_{\gls{sub:23}} \dot{h}_{\gls{sub:23}} + \gls{Qdot}_\text{r,2}
\end{equation}



\begin{equation}
\frac{d \gls{rhobar}_{\gls{sub:2}}}{d t} = ( \frac{\partial \rho_{\gls{sub:v}}}{\partial p_{\gls{sub:2}}} \cdot \bar{\alpha} +  \frac{\partial \rho_{\gls{sub:l}}}{\partial p_{\gls{sub:2}}} \cdot  (1 - \bar{\alpha})) \cdot  \frac{d p_{\gls{sub:2}}}{d t} + (\rho_{\gls{sub:v}} - \rho_{\gls{sub:l}}) \cdot [\frac{\partial \bar{\alpha}}{\partial p_{\gls{sub:2}}} \cdot  \frac{d p_{\gls{sub:2}}}{d t} + \frac{\partial \bar{\alpha}}{\partial h_\text{OUT,alpha}} \cdot \frac{d h_\text{OUT,alpha}}{d t} ]
\end{equation}


\begin{equation} \begin{split}
 \frac{d \gls{rhobar}_{\gls{sub:2}} \gls{hbar}_{\gls{sub:2}}}{d t} &=  [\bar{\alpha} \cdot ( \frac{\partial \rho_{\gls{sub:v}}}{\partial p_{\gls{sub:2}}} \cdot h_{\gls{sub:v}} + \frac{\partial h_{\gls{sub:v}}}{\partial p_{\gls{sub:2}}} \cdot \rho_{\gls{sub:v}} )  + ( 1- \bar{\alpha}) \cdot (  \frac{\partial \rho_{\gls{sub:l}}}{\partial p_{\gls{sub:2}}} \cdot h_{\gls{sub:l}} + \frac{\partial h_{\gls{sub:l}}}{\partial p_{\gls{sub:2}}} \cdot \rho_{\gls{sub:l}} )] \cdot \frac{d p_{\gls{sub:2}}}{d t} \\+
 &(\rho_{\gls{sub:v}}h_{\gls{sub:v}} - \rho_{\gls{sub:l}}h_{\gls{sub:l}}) \cdot  [\frac{\partial \bar{\alpha}}{\partial p_{\gls{sub:2}}} \cdot  \frac{d p_{\gls{sub:2}}}{d t} + \frac{\partial \bar{\alpha}}{\partial h_\text{OUT,alpha}} \cdot \frac{d h_\text{OUT,alpha}}{d t} ]
\end{split}
\end{equation}\\

\begin{center}
\textit{Constitutive equations}
\end{center}
\begin{equation}
\gls{rhobar}_{\gls{sub:2}} = \rho_{\gls{sub:v}} \bar{\alpha} + \rho_{\gls{sub:l}} \cdot (1 - \bar{\alpha})
\end{equation}

\begin{equation}
\gls{rhobar}_{\gls{sub:2}} \gls{hbar}_{\gls{sub:2}}= \rho_{\gls{sub:v}} h_{\gls{sub:v}} \bar{\alpha} + \rho_{\gls{sub:l}} h_{\gls{sub:l}} \cdot (1 - \bar{\alpha})
\end{equation}

\begin{equation}
\rho_{\gls{sub:v}} = dewDensity(sat)
\end{equation}
\begin{equation}
h_{\gls{sub:v}} = dewEnthalpy(sat)
\end{equation}
\begin{equation}
\frac{\partial \gls{hbar}_{\gls{sub:v}}}{\partial p_{\gls{sub:2}}}= dDewEnthalpy\_dPressure(sat)
\end{equation}

\begin{equation}
\bar{T}_{\gls{sub:2}} = temperature(sat)
\end{equation}
\begin{equation}
\gls{Qdot}_\text{r,2} = \pi \text{D} \gls{L}_{\gls{sub:2}} U_\text{pf,2} (T_\text{w,2} - \bar{T}_{\gls{sub:2}})
\end{equation}


%%%%%%%%%%%%%%%%%%%%%%%%%%%%%%%%% ZONE 2  METAL WALL %%%%%%%%%%%%%%%%%%%%%%%%%%%%%%%
\begin{flushleft}
{\bf Metal Wall}\\
\end{flushleft}
\begin{center}
\textit{Conservation equation}
\end{center}
Energy balance:\\
\begin{equation}
C_\text{w} \cdot \frac{M_\text{tot}}{\gls{L}} \cdot [\gls{L}_{\gls{sub:2}} \cdot  \frac{d \text{T}_\text{w,2}}{d t} +  (\text{T}_\text{w,12} - \text{T}_\text{w,23})  \frac{d \gls{L}_{\gls{sub:1}}}{d t} +  (\bar{\text{T}}_\text{w,2} - \text{T}_\text{w,23})  \frac{d \gls{L}_{\gls{sub:2}}}{d t}] = \gls{Qdot}_\text{sf,2} - \gls{Qdot}_\text{r,2}
\end{equation}



%%%%%%%%%%%%%%%%%%%%%%%%%%%%%%%%%%%%%%%%%%%%%%%%%%%%%%%%%%%%%%%%%%%%%%%%%%%%%
%%%%%%%%%%%%%%%%%%%%%%%%%%%%%% SUPER - HEATED ZONE (ZONE 3) %%%%%%%%%%%%%%%%%%%%%%%%%%%%%%%%%%%%
%%%%%%%%%%%%%%%%%%%%%%%%%%%%%%%%%%%%%%%%%%%%%%%%%%%%%%%%%%%%%%%%%%%%%%%%%%%%%
\begin{center}
{\bf SUPER-HEATED ZONE}
\end{center}


{\bf Primary fluid}\\
\begin{center}
\textit{Conservation equations}\\
\end{center}
Mass balance:
\begin{equation}
\text{A} [\gls{L}_{\gls{sub:3}}  \cdot \frac{d \gls{rhobar}_{\gls{sub:3}}}{d t} + (\rho_{\gls{sub:v}} - \gls{rhobar}_{\gls{sub:3}}) \cdot (\frac{d \gls{L}_{\gls{sub:1}}}{d t} + \frac{d \gls{L}_{\gls{sub:2}}}{d t})] = \gls{mdot}_{\gls{sub:23}} -  \gls{mdot}_{\gls{sub:out}}
\end{equation}
\begin{flushleft}
Energy balance:
\end{flushleft}
\begin{equation}
\text{A}\gls{L}_{\gls{sub:3}}[\gls{rhobar}_{\gls{sub:3}} \cdot \frac{d \gls{hbar}_{\gls{sub:3}}}{d t} + \gls{hbar}_{\gls{sub:3}} \cdot \frac{d \gls{rhobar}_{\gls{sub:3}}}{d t}  -  \frac{d p_{\gls{sub:3}}}{d t}] + \text{A}(\rho_{\gls{sub:v}}h_{\gls{sub:v}} - \gls{rhobar}_{\gls{sub:3}}\gls{hbar}_{\gls{sub:3}})\cdot (\frac{d \gls{L}_{\gls{sub:1}}}{d t} + \frac{d \gls{L}_{\gls{sub:2}}}{d t} ) = \gls{mdot}_{\gls{sub:23}}  \dot{h}_{\gls{sub:v}} -  \gls{mdot}_{\gls{sub:out}} \dot{h}_{\gls{sub:out}} + \gls{Qdot}_\text{r,3}
\end{equation}


\begin{equation}
\frac{d \gls{rhobar}_{\gls{sub:3}}}{d t} = [ \frac{\partial \gls{rhobar}_{\gls{sub:3}}}{\partial p_{\gls{sub:3}}} + \frac{1}{2} \cdot \frac{\partial \gls{rhobar}_{\gls{sub:3}}}{\partial \gls{hbar}_{\gls{sub:3}}} \cdot \frac{\partial \gls{hbar}_{\gls{sub:v}}}{\partial p_{\gls{sub:3}}}] \frac{d p_{\gls{sub:3}}}{d t} + \frac{1}{2} \cdot \frac{\partial \gls{rhobar}_{\gls{sub:3}}}{\partial \gls{hbar}_{\gls{sub:3}}}  \cdot \frac{d h_{\gls{sub:out}}}{d t}
\end{equation}


\begin{equation}
\frac{d \gls{hbar}_{\gls{sub:3}}}{d t} = \frac{1}{2} \cdot [\frac{\partial \gls{hbar}_{\gls{sub:v}}}{\partial p_{\gls{sub:3}}} \cdot \frac{d p_{\gls{sub:3}}}{d t} + \frac{d h_{\gls{sub:out}}}{d t}]
\end{equation}\\

\begin{center}
\textit{Constitutive equations}\\
\end{center}

\begin{equation}
vap = setState(p_{\gls{sub:3}},\gls{hbar}_{\gls{sub:3}})
\end{equation}
\begin{equation}
\gls{hbar}_{\gls{sub:3}} = \frac{1}{2}(h_{\gls{sub:v}} - h_{\gls{sub:out}})
\end{equation}
\begin{equation}
\frac{\partial \gls{rhobar}_{\gls{sub:3}}}{\partial \gls{hbar}_{\gls{sub:3}}}= density\_derh\_p(vap)
\end{equation}
\begin{equation}
\frac{\partial \gls{rhobar}_{\gls{sub:3}}}{\partial p_{\gls{sub:3}}}= density\_derp\_h(vap)
\end{equation}
\begin{equation}
\bar{T}_{\gls{sub:3}} = temperature(vap)
\end{equation}
\begin{equation}
\gls{Qdot}_\text{r,3} = \pi \text{D} \gls{L}_{\gls{sub:3}} U_\text{pf,3} (T_\text{w,3} - \bar{T}_{\gls{sub:3}})
\end{equation}

%%%%%%%%%%%%%%%%%%%%%%%%%%%%%%%% ZONE 3 METAL WALL  %%%%%%%%%%%%%%%%%%%%%%%%%%%%%%%%%%%%%%%%%%%%%%
{\bf Metal wall}\\
\begin{center}
\textit{Conservation equations}
\end{center}
Energy balance:
\begin{equation}
C_\text{w}\frac{M_\text{tot}}{L} \cdot [ \gls{L}_{\gls{sub:3}} \cdot  \frac{d \text{T}_\text{w,3}}{d t} + (\text{T}_\text{w,23} - \text{T}_\text{w,3})  (\frac{d \gls{L}_{\gls{sub:1}}}{d t} + \frac{d \gls{L}_{\gls{sub:2}}}{d t})]= \gls{Qdot}_\text{sf,3} - \gls{Qdot}_\text{r,3}
\end{equation}

\begin{center}
\textit{Constitutive equations}
\end{center}

\begin{equation}
\text{T}_\text{w,23} = \frac{   \text{T}_\text{w,3} \gls{L}_{\gls{sub:2}}  + \text{T}_\text{w,2}\gls{L}_{\gls{sub:3}}      }{  \gls{L}_{\gls{sub:2}} + \gls{L}_{\gls{sub:3}}         } 
\end{equation}



%%%%%%%%%%%%%%%%%%%%%%%%%%%%%%%%%%%%%%%%%%%%%%%%%%%%%%%%%%%%%%%%%%%%%%%%%%%%%%%%%%%%%%%
%%%%%%%%%%%%%%%%%%%%%%%%%%%%%%%%% HEAT TRANSFER FORMULATION %%%%%%%%%%%%%%%%%%%%%%%%%%%%%%%%%%%%%%%
%%%%%%%%%%%%%%%%%%%%%%%%%%%%%%%%%%%%%%%%%%%%%%%%%%%%%%%%%%%%%%%%%%%%%%%%%%%%%%%%%%%%%%%%%
\subsection{Heat Transfer}

Based on \gls{Nu} from \gls{Rey} and \gls{Pra} for a characteristic length \gls{L}. Angles are usually calculated in radians or \gls{pi}.


%%%%%%%%%%%%%%%%%%%%%%%%%%%%%%%%%%%%%%%%%%%%%%%%%%%%%%%%%%%%%%%%%%%%%%%%%%%%%%%%%%%%%%%
%%%%%%%%%%%%%%%%%%%%%%%%%%%%%%%%% Pressure Drop FORMULATION %%%%%%%%%%%%%%%%%%%%%%%%%%%%%%%%%%%%%%%
%%%%%%%%%%%%%%%%%%%%%%%%%%%%%%%%%%%%%%%%%%%%%%%%%%%%%%%%%%%%%%%%%%%%%%%%%%%%%%%%%%%%%%%%%
\subsection{Pressure Drop}



%%%%%%%%%%%%%%%%%%%%%%%%%%%%%%%%%%%%%%%%%%%%%%%%%%%%%%%%%%%%%%%%%%%%%%%%%%%%%%%%%%%%%%%
%%%%%%%%%%%%%%%%%%%%%%%%%%%%%%%%% Results and Discussion %%%%%%%%%%%%%%%%%%%%%%%%%%%%%%%%%%%%%%%
%%%%%%%%%%%%%%%%%%%%%%%%%%%%%%%%%%%%%%%%%%%%%%%%%%%%%%%%%%%%%%%%%%%%%%%%%%%%%%%%%%%%%%%%%
\section{Results and Discussion}

Compared to \cite{Kaern2011b}, the model .... 


\cite{Zhang2006}

\cite{Zhang2009}

\section{Conclusion}





%%%%%%%%%%%%%%%%%%%%%%%%%%%%%%%%%%%%%%%%%%%%%%%%%%%%%%%%%%%%%%%%%%%%%%%%%%%%%%%%%%%%%%%%%%%
%%%%%%%%%%%%%%%%%%%%%%%%%% SECONDARY FLUID %%%%%%%%%%%%%%%%%%%%%%%%%%%%%%%%%%%%%%%%%%%%%%%%%%%%%%%%
%%%%%%%%%%%%%%%%%%%%%%%%%%%%%%%%%%%%%%%%%%%%%%%%%%%%%%%%%%%%%%%%%%%%%%%%%%%%%%%%%%%%%%%%%%%%
\subsection{Secondary fluid}

The secondary fluid is treated as a constant heat capacity fluid. Two different method to solve the heat exchanger are developed. Newton law of cooling and epsilon-NTU.

%%%%%%%%%%%%%%%%%%%%%%%%%%%%%%%%%%%%%%%%%%%%%%%%%%%%%%
%%%%%%%%%%%%%% MODELING OF THE MOVING BOUNDARY MODEL IN MODELICA %%%%%%%%%%%
%%%%%%%%%%%%%%%%%%%%%%%%%%%%%%%%%%%%%%%%%%%%%%%%%%%%%%

\section{Modelling}

\subsection{Modelica Characteristics}

\begin{itemize}
\item Enthalpy controls switching
\item Only length gets defined, no zones get disabled. Their respective length might be set to \num{e-10} though. 
\item Solid wall with constant properties is included in the cell of the primary fluid. 
\item Should length $L$ be treated as input, output or as a variable in the connectors?
\item Implementation of heat transfer and void fractions models starting with \cite{Kaern2011b} and Sylvain's EES file.
\end{itemize}


%%%%%%%%%%%%%%%%%%%%%%%%%%%%%%%%%%%%%%%%%%%%%%%%
%%%%%%%%%%%%%%%%%% OLD SECONDARY FLUID FORMULATION %%%%%%%%%%%%%%%%%%%%%%
%%%%%%%%%%%%%%%%%%%%%%%%%%%%%%%%%%%%%%%%%%%%%%%%

\include{OldSecondaryFluid}


%%%%%%%%%%%%%%%%%%%%%%%%%%%%%%%%%%%%%%%%%%%%%%%%
%%%%%%%%%%%%%%%%%% BIBLIOGRAPHY %%%%%%%%%%%%%%%%%%%%%%
%%%%%%%%%%%%%%%%%%%%%%%%%%%%%%%%%%%%%%%%%%%%%%%%


\bibliographystyle{plain}
\bibliography{references}


%%%%%%%%%%%%%%%%%%%%%%%%%%%%%%%%%%%%%%%%%%%%%%%%
%%%%%%%%%%%%%%%%%% NOMENCLATURE %%%%%%%%%%%%%%%%%%%%%%
%%%%%%%%%%%%%%%%%%%%%%%%%%%%%%%%%%%%%%%%%%%%%%%%


%\section{Examples for Nomenclature}
%When used the first time, the full description appears for acronyms: \gls{example_ddye}, \gls{example_R0}, \gls{example_r0}, \gls{example_kdeac}\\
%
%Plural versions can be printed by \glspl{example_r0} and \glspl{example_kdeac}
%Capital letters for sentence beginnings are available as well \Gls{example_ddye}
%
%.. and now we also use the symbols \gls{example_pi}.

%\begin{multicols}{2}
\setglossarysection{section} 
\printglossary[type=main]
\setglossarysection{subsection} 
\printglossary[type=acronym]
\printglossary[type=dimlessnumber]
\printglossary[type=greeksymbols]
\printglossary[type=subscripts]
%\end{multicols}

%\section{Nomenclature}
%\printindex
%\printglossaries




\end{document}
