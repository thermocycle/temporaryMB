% !TEX TS-program = pdflatex
% !TEX encoding = UTF-8 Unicode

% This is a simple template for a LaTeX document using the "article" class.
% See "book", "report", "letter" for other types of document.

\documentclass[11pt]{article} % use larger type; default would be 10pt

%\usepackage[utf8]{inputenc} % set input encoding (not needed with XeLaTeX)
\usepackage[latin1]{inputenc}

%%% Examples of Article customizations
% These packages are optional, depending whether you want the features they provide.
% See the LaTeX Companion or other references for full information.

%%% PAGE DIMENSIONS
\usepackage{geometry} % to change the page dimensions
\geometry{a4paper} % or letterpaper (US) or a5paper or....
% \geometry{margin=2in} % for example, change the margins to 2 inches all round
% \geometry{landscape} % set up the page for landscape
%   read geometry.pdf for detailed page layout information

\usepackage{graphicx} % support the \includegraphics command and options

% \usepackage[parfill]{parskip} % Activate to begin paragraphs with an empty line rather than an indent

%%% PACKAGES
\usepackage{booktabs} % for much better looking tables
\usepackage{array} % for better arrays (eg matrices) in maths
\usepackage{paralist} % very flexible & customisable lists (eg. enumerate/itemize, etc.)
\usepackage{verbatim} % adds environment for commenting out blocks of text & for better verbatim
\usepackage{subfig} % make it possible to include more than one captioned figure/table in a single float
% These packages are all incorporated in the memoir class to one degree or another...

%%% HEADERS & FOOTERS
\usepackage{fancyhdr} % This should be set AFTER setting up the page geometry
\pagestyle{fancy} % options: empty , plain , fancy
\renewcommand{\headrulewidth}{0pt} % customise the layout...
\lhead{}\chead{}\rhead{}
\lfoot{}\cfoot{\thepage}\rfoot{}

%%% SECTION TITLE APPEARANCE
\usepackage{sectsty}
\allsectionsfont{\sffamily\mdseries\upshape} % (See the fntguide.pdf for font help)
% (This matches ConTeXt defaults)

%%% ToC (table of contents) APPEARANCE
\usepackage[nottoc,notlof,notlot]{tocbibind} % Put the bibliography in the ToC
\usepackage[titles,subfigure]{tocloft} % Alter the style of the Table of Contents
\renewcommand{\cftsecfont}{\rmfamily\mdseries\upshape}
\renewcommand{\cftsecpagefont}{\rmfamily\mdseries\upshape} % No bold!

%%% END Article customizations
\usepackage{hyperref,siunitx}
%\usepackage{natbib}
\usepackage[
    %,style=altlist
    %,hypertoc=true
    %,hyper=true
    %,number=none
    ,acronym=true
    %,toc=true
    ,section=subsection
    ,xindy
]{glossaries}


\newglossary{dimlessnumber}{dimnu}{dimnb}{Dimensionless Numbers}
\newglossary{greeksymbols}{grsym}{grsbl}{Greek Symbols}
\newglossary{subscripts}{subsc}{subcr}{Subscripts}



\newglossary{example_acronym}{exacr}{excro}{Example Acronyms}
\newglossary{example_symbols}{exsym}{exsbl}{Example Greek Symbols}
\newglossary{example_glossary}{exglo}{exgls}{Example Glossary}


\makeglossaries
\usepackage[xindy]{imakeidx}
\makeindex


%%% The "real" document content comes below...

\title{ThermoCycle Moving Boundary Model}
\author{Adriano Desideri\\
\small Thermodynamics Laboratory\\[-0.8ex]
\small University of Li\`ege\\[-0.8ex]
\small Li\`ege, Belgium\\
\small \texttt{adesideri@ulg.ac.be}\\
\and
Jorrit Wronski\\
\small Department of Mechanical Engineering\\[-0.8ex]
\small Technical University of Denmark\\[-0.8ex]
\small Kgs. Lyngby, Denmark\\
\small \texttt{jowr@mek.dtu.dk}
}

%\date{} % Activate to display a given date or no date (if empty),
         % otherwise the current date is printed 

\begin{document}

% %%%%%%%%%%%%%%%%%%%%%%%%%%%%%%%%%%%%%%%%%%%%%%%%%%%%%%%%%%%%%%%%%%%%%%%
% Dimensionless Groups (based on http://robert.mathmos.net/research/latex/rjwmath.sty)
%  (All two letter command sequences, except for those names already
%   used by LaTeX, and where we need to avoid ambiguity)
% %%%%%%%%%%%%%%%%%%%%%%%%%%%%%%%%%%%%%%%%%%%%%%%%%%%%%%%%%%%%%%%%%%%%%%%
\newcommand{\Bo}{\ensuremath{\operatorname{Bo}}}  % Bond
\newcommand{\Ca}{\ensuremath{\operatorname{Ca}}}  % Capillary
\newcommand{\De}{\ensuremath{\operatorname{De}}}  % Deborah
\newcommand{\Ek}{\ensuremath{\operatorname{E}}}   % Ekman
\newcommand{\Fr}{\ensuremath{\operatorname{Fr}}}  % Froude
\newcommand{\Gr}{\ensuremath{\operatorname{Gr}}}  % Grashof
\newcommand{\Le}{\ensuremath{\operatorname{Le}}}  % Lewis
\newcommand{\Ma}{\ensuremath{\operatorname{Ma}}}  % Mach
\newcommand{\Nu}{\ensuremath{\operatorname{Nu}}}  % Nusselt
\newcommand{\Pe}{\ensuremath{\operatorname{Pe}}}  % Peclet
\newcommand{\Pra}{\ensuremath{\operatorname{Pr}}} % Prandtl
\newcommand{\Ra}{\ensuremath{\operatorname{Ra}}}  % Rayleigh
\newcommand{\Rey}{\ensuremath{\operatorname{Re}}} % Reynolds
\newcommand{\Ri}{\ensuremath{\operatorname{Ri}}}  % Richardson
\newcommand{\Ro}{\ensuremath{\operatorname{Ro}}}  % Rossby
\newcommand{\Sc}{\ensuremath{\operatorname{Sc}}}  % Schmidt
\newcommand{\Sta}{\ensuremath{\operatorname{Sta}}}% Stanton
\newcommand{\Ste}{\ensuremath{\operatorname{S}}}  % Stefan
\newcommand{\Str}{\ensuremath{\operatorname{St}}} % Strouhal
\newcommand{\Ta}{\ensuremath{\operatorname{Ta}}}  % Taylor
\newcommand{\We}{\ensuremath{\operatorname{We}}}  % Weber
\newcommand{\Wi}{\ensuremath{\operatorname{Wi}}}  % Weissenberg
\newcommand{\Wo}{\ensuremath{\operatorname{Wo}}}  % Womersley
%
\newcommand{\Gz}{\ensuremath{\operatorname{Gz}}}  % Graetz
% 
% %%%%%%%%%%%%%%%%%%%%%%%%%%%%%%%%%%%%%%%%%%%%%%%%%%%%%%%%%%%%%%%%%%%%%%%
% Other things
% %%%%%%%%%%%%%%%%%%%%%%%%%%%%%%%%%%%%%%%%%%%%%%%%%%%%%%%%%%%%%%%%%%%%%%%
% Upright pi command 
%\providecommand{\upi}{\mathrm{\pi}}
%
% %%%%%%%%%%%%%%%%%%%%%%%%%%%%%%%%%%%%%%%%%%%%%%%%%%%%%%%%%%%%%%%%%%%%%%%
% Overlap environments for math: http://math.arizona.edu/~aprl/publications/mathclap/
% %%%%%%%%%%%%%%%%%%%%%%%%%%%%%%%%%%%%%%%%%%%%%%%%%%%%%%%%%%%%%%%%%%%%%%%
% For comparison, the existing overlap macros:
% \def\llap#1{\hbox to 0pt{\hss#1}}
% \def\rlap#1{\hbox to 0pt{#1\hss}}
\def\clap#1{\hbox to 0pt{\hss#1\hss}}
\def\mathllap{\mathpalette\mathllapinternal}
\def\mathrlap{\mathpalette\mathrlapinternal}
\def\mathclap{\mathpalette\mathclapinternal}
\def\mathllapinternal#1#2{%
\llap{$\mathsurround=0pt#1{#2}$}}
\def\mathrlapinternal#1#2{%
\rlap{$\mathsurround=0pt#1{#2}$}}
\def\mathclapinternal#1#2{%
\clap{$\mathsurround=0pt#1{#2}$}}



% %%%%%%%%%%%%%%%%%%%%%%%%%%%%%%%%%%%%%%%%%%%%%%%%%%%%%%%%%%%%%%%%%%%%%%%
% Glossaries 
% Read this entry for details: http://www.latex-community.org/know-how/263-glossaries-nomenclature-lists-of-symbols-and-acronyms
% %%%%%%%%%%%%%%%%%%%%%%%%%%%%%%%%%%%%%%%%%%%%%%%%%%%%%%%%%%%%%%%%%%%%%%%

% First we define some entries for the normal glossary
\newglossaryentry{L}{name={\ensuremath{L}},description={Length (\si{\metre})}}

% Then we define some acronyms 
\newacronym{MB}{MB}{moving boundary}

% Next come the dimensionless numbers
\newglossaryentry{Nu}{type=dimlessnumber,
name={\Nu},
description={Nusselt number: $\Nu = \frac{\text{Convection}}{\text{Conduction}} = \frac{hL}{k_f}$},  
%text={\Nu},
first={Nusselt number (\Nu)},
plural={Nusselt numbers},
firstplural={Nusselt numbers (\Nu)}}
%
\newglossaryentry{Rey}{type=dimlessnumber,
name={\Rey},
description={Reynolds number},
%text={\Rey},
first={Reynolds number (\Rey)},
plural={Reynolds numbers},
firstplural={Reynolds numbers (\Rey)}}
%
\newglossaryentry{Pr}{type=dimlessnumber,
name={\Pr},
description={Prandtl number},
%text={\Pr},
first={Prandtl number (\Pr)},
plural={Prandtl numbers},
firstplural={Prandtl numbers (\Pr)}}
%
\newglossaryentry{gr}{type=dimlessnumber,
name={\Gr},
description={Grashof number},
%text={\Gr},
first={Grashof number (\Gr)},
plural={Grashof numbers},
firstplural={Grashof numbers (\Gr)}}
%
\newglossaryentry{ra}{type=dimlessnumber,
name={\Ra},
description={Rayleigh number},
%text={\Ra},
first={Rayleigh number (\Ra)},
plural={Rayleigh numbers},
firstplural={Rayleigh numbers (\Ra)}}
%
\newglossaryentry{gz}{type=dimlessnumber,
name={\Gz},
description={Graetz number},
%text={\Gz},
first={Graetz number (\Gz)},
plural={Graetz numbers},
firstplural={Graetz numbers (\Gz)}}

% And now the Greek symbols
\newglossaryentry{pi}{type=greeksymbols,name={\ensuremath{\pi}},sort=pi,description={ratio of circumference of circle to its diameter}}

% and subscripts
\newglossaryentry{r}{type=subscripts,name={r},description={refrigerant, working fluid}}


% Examples
\newacronym[type=example_acronym]{example_ddye}{D$_{\text{dye}}$}{donor dye, ex. Alexa 488}
\newacronym[type=example_acronym,description={\glslink{example_r0}{F\"{o}rster distance}}]{example_R0}{$R_{0}$}{F\"{o}rster distance}
\newglossaryentry{example_r0}{type=example_glossary,name=\glslink{example_R0}{\ensuremath{R_{0}}},text=F\"{o}rster distance,description={F\"{o}rster distance, where 50\% ...}, sort=R}
\newglossaryentry{example_kdeac}{type=example_glossary,name=\glslink{example_R0}{\ensuremath{k_{DEAC}}},text=$k_{DEAC}$, description={is the rate of deactivation from ... and emission)}, sort=k}
\newglossaryentry{example_pi}{type=example_symbols,name={\ensuremath{\pi}},sort=pi,description={ratio of circumference of circle to its diameter}}


\maketitle

%======================================================================
%      Glossary and acronyms
%======================================================================
%      \cleardoublepage
%      \chapter*{Lists of Symbols and \glossaryname}
%      Please note that the given units do not apply exclusively. In some cases, different units may be given close to the appearance of a symbol in text or calculation. 
%      \renewcommand{\glossarypreamble}{Numbers in italic indicate primary definitions.}
%      \printglossary[type=main]
%      \renewcommand{\glossarypreamble}{Acronyms}
%      \printglossary[type=greek]
%      \printglossary[type=roman]
%      \printglossary[type=script]
%      \printglossary[type=acronym]
%      \printglossary[type=symbolslist]
%      \printglossary[type=symbolslist,style=mylist]
%      \printglossary[type=characteristicnumbers,style=mylist]
%      \printglossaries
%      \printacronym

\abstract{Test}

\section{Motivation}


Your text goes here.

\Gls{MB} formulation based on \cite{Bendapudi2008}. 

\section{Formulation}
\subsection{Assumptions}

\subsection{Equations}

\subsection{Heat Transfer}

Based on \gls{Nu} for a characteristic length \gls{L}. Angles are usually calculated in radians or \gls{pi}.

\subsection{Pressure Drop}

\section{Results and Discussion}

Compared to \cite{Kaern2011b}, the model .... 


\cite{Zhang2006}

\cite{Zhang2009}

\section{Conclusion}

\bibliographystyle{plain}
\bibliography{references}

\section{Examples for Nomenclature}
When used the first time, the full description appears for acronyms: \gls{example_ddye}, \gls{example_R0}, \gls{example_r0}, \gls{example_kdeac}\\

Plural versions can be printed by \glspl{example_r0} and \glspl{example_kdeac}
Capital letters for sentence beginnings are available as well \Gls{example_ddye}

.. and now we also use the symbols \gls{example_pi}.

\setglossarysection{section} 
\printglossary[type=main]
\setglossarysection{subsection} 
\printglossary[type=acronym]
\printglossary[type=dimlessnumber]
\printglossary[type=greeksymbols]
\printglossary[type=subscripts]

%\section{Nomenclature}
%\printindex
%\printglossaries

\end{document}
